\sectionwithlogo{Общие сведения}{\includeMPgraphics{figure-clock}}

%%%%%%%%%%%%%%%%%%%%%%%%%

\begin{frame}{Что такое \hologo{METAPOST}?}
\begin{itemize}
\item
это алгоритмический язык
\item
это система программирования (компилятор~+~библиотеки)
\end{itemize}

\hologo{METAPOST} был создал Джоном Хобби (John~D.~Hobby).

Во многом система \hologo{METAPOST} основана на системе \hologo{METAFONT},
созданной известным американским математиком Дональдом Кнутом
(Donald~E.~Knuth). \hologo{METAFONT} задуман как средство изготовления шрифтов
для системы компьютерной вёрстки \TeX.
\end{frame}

%%%%%%%%%%%%%%%%%%%%%%%%%

\begin{frame}{Возможности \hologo{METAPOST}}

\begin{itemize}
\item
создание рисунков в~векторном графическом формате PostScript

\item
сложные вычисления при подготовке рисунков

\item
широкий набор встроенных и~библиотечных команд

\item
возможность программирования пользовательских команд

\item
вставка тескта, отформатированного в~системе~\TeX

\end{itemize}
\end{frame}

%%%%%%%%%%%%%%%%%%%%%%%%%

\begin{frame}{Сфера применения \hologo{METAPOST}}

\begin{columns}

\column{.55\textwidth}
\begin{itemize}
\item
\alert<1>{подготовка чертежей и~технических рисунков}
\item
\alert<2>{диаграмм}
\item
\alert<3>{графиков функций}
\item
\alert<4>{иллюстраций по геометрии (с~формулами)}
\item
\alert<5>{логотипов}
\end{itemize}

\column{.45\textwidth}
\only<1>{\includeMPgraphics{figure-clock}}
\only<2>{\includeMPgraphics{figure-piechart}}
\only<3>{\includeMPgraphics{figure-graph}}
\only<4>{\includeMPgraphics{figure-fejer}}
\only<5>{\includeMPgraphics{figure-mmflogo}}

\end{columns}

\transdissolve<2->

\end{frame}

%%%%%%%%%%%%%%%%%%%%%%%%%

\begin{frame}{Особенности \hologo{METAPOST}}

\hologo{METAPOST} \alert{не} является системой WYSIWYG (What You See Is What
You Get~— «вы получаете то, что видите»). Это значит, что рисунок готовится как
программа, которая компилируется, и~в~результате компиляции получается файл
в~формате PostScript, содержащий изображение.

\end{frame}

%%%%%%%%%%%%%%%%%%%%%%%%%

\begin{frame}{PostScript}

PostScript~— это не только графический формат, но и~алгоритмический язык. В~нём
заложены богатые графические и~вычислительные возможности.

Однако из-за особенностей этого языка на нём трудно программировать «вручную».
Такое программирование требует от программиста совершенно иного мышления,
нежели программирование на «традиционных» алгоритмических языках.

Чаще всего программы на PostScript генерируются автоматически другими
программами.

\end{frame}

%%%%%%%%%%%%%%%%%%%%%%%%%

\begin{frame}{Предназначение PostScript}
Главное предназначение PostScript~— управление растровыми устройствами, такими
как принтеры, экран монитора или графический файл.

Некоторые устройства вроде дорогих профессиональных принтеров имеют встроенный
интерпретатор PostScript. Другие понимают более примитивный управляющий язык,
поэтому для управления ими требуется программа"=переводчик (транслятор).
Наиболее популярный транслятор языка PostScript~— программа GhostScript.
\end{frame}

%%%%%%%%%%%%%%%%%%%%%%%%%

\begin{frame}{Пример на языке PostScript}
\begin{columns}
\column{.3\textwidth}
\includegraphics{pstest}
\column{.7\textwidth}
\begin{programlisting}
\%!PS-Adobe\par
\leavevmode\par
5 setlinewidth\par
1 .388235294117647 .27843137254902\par
~~~~setrgbcolor\par
\leavevmode\par
newpath\par
8 dup moveto\par
72 dup rlineto\par
-72 0 rlineto\par
72 -72 rlineto\par
stroke\par
\leavevmode\par
showpage
\end{programlisting}
\end{columns}

\end{frame}

%%%%%%%%%%%%%%%%%%%%%%%%%

\begin{frame}{Где взять \hologo{METAPOST}}
\begin{itemize}
\item
\hologo{METAPOST} для Linux, Windows, Mac~OS~X входит в~пакет программ
\TeX~Live:
\url{https://tug.org/texlive/}
\item
\hologo{METAPOST} для Windows входит в~пакет программ MiK\TeX:
\url{https://miktex.org/}
\end{itemize}
\end{frame}

\begin{frame}
\frametitle{Что почитать про \hologo{METAPOST}\only<2->{
	{\mdseries\itshape(продолжение)}}}
\begin{thebibliography}{1}
\small
\only<1>{
\bibitem{bib:Hobby}
A User’s Manual for \hologo{METAPOST}
\newblock
Hobby, John~D.
\newblock
\url{https://www.tug.org/docs/metapost/mpman.pdf}

\bibitem{bib:Hobby}
\hologo{METAPOST}. Руководство пользователя
\newblock
Hobby, John D.
\newblock
\url{http://www.ctan.org/tex-archive/info/metapost/doc/russian/mpman-ru/mpman-ru.pdf}

\bibitem{bib:Grogono}
\hologo{METAPOST}: A~Reference Manual
\newblock
Grogono, Peter
\newblock
\url{http://users.encs.concordia.ca/~grogono/Writings/mpref.pdf}

\bibitem{bib:Wiki}
\hologo{METAPOST} — Википедия
\newblock
\url{http://ru.wikipedia.org/wiki/MetaPost}
\newblock
Статья в русской Википедии

\bibitem{bib:Zoonekynd}
Exemples d’utilisation de Métapost
\newblock
Zoonekynd, Vincent
\newblock
\url{http://zoonek.free.fr/LaTeX/Metapost/metapost.html}

}

\only<2>{%
\bibitem{bib:Baldin}
Создание иллюстраций в~\hologo{METAPOST}
\newblock
Балдин, Е. М.
\newblock
\url{http://www.inp.nsk.su/~baldin/mpost/}
\newblock
Статья для журнала «Linux Format»

\bibitem{bib:Heck}
Learning \hologo{METAPOST} by Doing
\newblock
Heck, André
\newblock
\url{http://remote.science.uva.nl/~heck/Courses/mptut.pdf}

\bibitem{bib:Urs}
\hologo{METAPOST}: A Very Brief Tutorial
\newblock
Urs, Oswald
\newblock
\url{http://www.ursoswald.ch/metapost/tutorial.pdf}

\bibitem{bib:Hurlin}
Practical introduction to~\hologo{METAPOST}
\newblock
Hurlin, Clément
\newblock
\url{http://www-sop.inria.fr/everest/Clement.Hurlin/misc/Practical-introduction-to-MetaPost.pdf}
}

\only<3>{%
\bibitem{bib:Thurston}
Drawing with \hologo{METAPOST}
\newblock
Thurston, Toby
\newblock
\url{https://raw.githubusercontent.com/thruston/Drawing-with-Metapost/master/Drawing-with-Metapost.pdf}

\bibitem{bib:Knuth:TheMETAFONTbookWilliams}
Всё про \hologo{METAFONT}
\newblock
Кнут, Дональд Э.
\newblock
Москва~: Вильямс, 2003.~— ISBN 5-8459-0442-0 (рус.)
}
\end{thebibliography}
\end{frame}
