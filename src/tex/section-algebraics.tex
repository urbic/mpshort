\sectionwithlogo{Алгебраические выражения}{}

\begin{frame}{Арифметические операторы}
\begin{center}
\begin{tabular}{|c|c|}
\hline
\literal{+\,}$x$&$+x$\\\hline
\literal{-\,}$x$&$-x$\\\hline
$x$\literal{\,+\,}$y$&$x+y$\\\hline
$x$\literal{\,-\,}$y$&$x-y$\\\hline
$x$\literal{\,*\,}$y$&$xy$\\\hline
$x$\literal{\,/\,}$y$&$x\fracslash y$\\\hline
$x$\literal{\,**\,}$y$&$x^y$\\\hline
$x$\literal{\,++\,}$y$&$\sqrt{x^2+y^2}$\\\hline
$x$\literal{\,+-+\,}$y$&$\sqrt{x^2-y^2}$\\\hline
\literal{sqrt~}$x$&$\sqrt x$\\\hline
\literal{abs~}$x$&$|x|$\\\hline
\literal{floor~}$x$&целая часть~$x$\\\hline
\literal{round~}$x$&округление~$x$\\\hline
$x$\literal{~mod~}$y$&остаток от деления~$x$ на~$y$\\\hline
$x$\literal{~div~}$y$&целая часть отношения~$x$ и~$y$\\\hline
\end{tabular}
\end{center}
\end{frame}

\begin{frame}{Тригонометрия}
\begin{center}
\begin{tabular}{|c|c|}
\hline
\literal{cosd~}$x$&$\cos x^\circ$\\\hline
\literal{sind~}$x$&$\sin x^\circ$\\\hline
\end{tabular}
\end{center}
\end{frame}


\begin{frame}{Медиатор}
Для числа~$t$ и~величин $a$~и~$b$ одинакового типа, либо \literal{numeric},
либо \literal{pair}, либо \literal{color}, в~\hologo{METAPOST} определяется
операция \emph{медиатор}:

\begin{center}
\LARGE $t$\literal[$a$\literal{,~}$b$\literal]\quad$\equiv$\quad$(1-t)\,a+t\,b$
\end{center}
\end{frame}

\begin{frame}{Медиатор}
\begin{center}
\LARGE
\literal{.2[3, 8]}\quad$\equiv$\quad\literal4\\[3ex]
\literal{.6[z1, z2]}\\
$\equiv$\\
{\normalsize точка, делящая отрезок от \literal{z1} до \literal{z2}
в~отношении~$3:2$}
\end{center}
\end{frame}

\begin{frame}{Кинематический смысл медиатора}
\begin{columns}
\column{.3\textwidth}
\leavevmode
\begin{mplibcode}
beginfig(0)

z1=origin;
z2=(1cm, 3.5cm);
z3=(-.5)[z1, z2];
z4=1.5[z1, z2];
z5=(2cm, -cm);

draw z3--z4 withpen thinpen dashed evenly scaled .5;
draw z1--z2 withpen boldpen;

markdot.circle(z1) fg=>orange;
markdot.circle(z2) fg=>orange;

for i:=-.4 step .1 until 1.5:
	drawarrow z5-- .975[z5, i[z1, z2]] withpen thinpen withcolor red;
endfor

markdot.circle(z5) fg=>orange;

label.lft("$z_0$", z1);
label.lft("$z_1$", z2);

endfig
\end{mplibcode}

\column{.7\textwidth}
Между прочим, выражение
	\[
	z(t)=(1-t)z_0+tz_1
	\]
задаёт закон равномерного прямолинейного движения, при котором точка
с~радиусом"=вектором~$z(t)$ в~момент времени~$t=0$ проходит через точку~$z_0$,
а~при~$t=1$~— через~$z_1$.

\bigskip
Отрицательные значения~$t$ соответствуют точкам, находящимся на продолжении
отрезка~$z_0z_1$ за точку~$z_0$, а~значения, бо́льшие единицы, отвечают точкам
на продолжении этого отрезка за точку~$z_1$.
\end{columns}
\end{frame}
