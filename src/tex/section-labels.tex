\sectionwithlogo{Надписи}
	{\includeMPgraphics{figure-metapost-shadowed}}

\begin{frame}{Подготовка надписей}
\hologo{METAPOST} позволяет размещать на~картинке надписи.

\hologo{METAPOST} может использовать внешние программы для подготовки
надписей~— troff и~\TeX.

Изучение всех возможностей систем troff и~\TeX\ не входит в~наши планы. Мы
совершенно обойдём вниманием troff, но обсудим простейшие и~важнейшие
возможности~\TeX.
\end{frame}

\begin{frame}{Введение в~\TeX}
\TeX~— система компьютерной вёрстки, созданная уже упомянутым
профессором Кнутом.

\alert{\TeX\ произносится отнюдь не как [текс], а~как [тех].}

Особенностью~\TeX\ является превосходная поддержка набора математических
формул.

Между прочим, это руководство подготовлено в~\TeX.
\end{frame}

\begin{frame}{Назначение и~особенности~\TeX}
Назначение системы~\TeX\ примерно то~же самое, что и~у~известных программ
Microsoft Word, OpenOffice Writer, Adobe Pagemaker, QuarkXPress и~подобных.

Однако, подобно системе~\hologo{METAPOST}, \TeX\ \alert{не} является системой
WYSIWYG. Будущая вёрстка готовится как программа, в~которой обычный текст
перемежается с~управляющими командами (кстати, макрокомандами), которые влияют
на вид документа. Для получения документа, который можно печатать, требуется
компиляция.

Это обстоятельство несколько затрудняет работу, однако именно оно является
причиной великолепного полиграфического качества документов, подготовленных
в~\TeX, и~стилевого их единообразия.
\end{frame}

\begin{frame}{Простейшие \TeX{}нические решения}
\begin{large}
\begin{tabular}{rl}
\literal{Просто текст}
&\rmfamily Просто текст
\\[1ex]
\literal{\textbackslash textit\{Курсивный\} текст}
&\rmfamily\textit{Курсивный} текст
\\[1ex]
\literal{\textbackslash textbf\{Жирный\} текст}
&\rmfamily\textbf{Жирный} текст
\\[1ex]
\literal{\textbackslash textit\{\textbackslash textbf\{Жирный курсив\}\}}
&\rmfamily\textit{\textbf{Жирный курсив}}
\\[1ex]
\literal{\textbackslash texttt\{Моноширинный\} текст}
&\rmfamily\texttt{Моноширинный} текст
\\[1ex]
\literal{Формула \$x\$}
&\rmfamily Формула $x$
\end{tabular}
\end{large}
\end{frame}

\begin{frame}{Простейшие \TeX{}нические решения}
\begin{large}
\begin{tabular}{rl}
\literal{\$a+b\$}
&$a+b$
\\[1ex]
\literal{\$a-b=2\$}
&$a-b=2$
\\[1ex]
\literal{\$\textbackslash frac12\$}
&$\frac12$
\\[1ex]
\literal{\$\textbackslash frac1\{12\}\$}
&$\frac1{12}$
\\[1ex]
\literal{\$\textbackslash frac\{123\}\{456\}\$}
&$\frac{123}{456}$
\\[1ex]
\literal{\$\textbackslash sqrt2\$}
&$\sqrt2$
\\[1ex]
\literal{\$\textbackslash root3\textbackslash of2\$}
&$\sqrt[3]2$
\\[1ex]
\literal{\$\textbackslash alpha\textbackslash beta\textbackslash
gamma\textbackslash ldots\textbackslash omega\$}
&$\alpha\beta\gamma\ldots\omega$
\\[1ex]
\literal{\$z\_2\$}
&$z_2$
\\[1ex]
\literal{\$z\textasciicircum\{12\}\$}
&$z^{12}$
\\[1ex]
\literal{\$\textbackslash cos\textasciicircum2\textbackslash
theta+\textbackslash sin\textasciicircum2\textbackslash theta=1\$}
&$\cos^2\theta+\sin^2\theta=1$
\end{tabular}
\end{large}
\end{frame}

\begin{frame}{Простейшие \TeX{}нические решения}
\begin{large}
\begin{tabular}{rl}
\literal{\$x\textbackslash in[-\textbackslash pi;\textbackslash pi)\$}
&$x\in[-\pi;\pi)$
\\[1ex]
\literal{\$x\textbackslash in\textbackslash\{2,3,5,7,11,13,17,\textbackslash ldots\textbackslash\}\$}
&$x\in\{2,3,5,7,11,13,17,\ldots\}$
\\[1ex]
\literal{\$\textbackslash mathbb N\textbackslash subset\textbackslash mathbb Z\$}
&$\mathbb N\subset\mathbb Z$
\\[1ex]
\literal{\$|\textbackslash tg x|>|\textbackslash
sin x|\$ при \$x>0\$}
&\rmfamily$\mathopen|\tg x\mathclose|>\mathopen|\sin x\mathclose|$ при $x>0$
\\[1ex]
\literal{\$\textbackslash sqrt\{ab\}\textbackslash leqslant\textbackslash
frac\{a+b\}2\$}
&$\sqrt{ab}\leqslant\frac{a+b}2$
\\[1ex]
\literal{\$\textbackslash sqrt\{ab\}\textbackslash geqslant\textbackslash
frac\{2ab\}\{a+b\}\$}
&$\sqrt{ab}\geqslant\frac{2ab}{a+b}$
\end{tabular}
\end{large}
\end{frame}

\begin{frame}{Более сложный пример: формула Коши}
\begin{Huge}
	\[
	\alert<2>{f(z)=}
		\alert<3>{\frac1{2\pi i}}
		\alert<4>{\oint\limits_{\partial U_\varepsilon(z)}}
		\alert<5>{\frac{f(\zeta)\,d\zeta}{\zeta-z}}
	\]
\end{Huge}

\begin{center}
\LARGE
\begin{programlisting}
\$\alert<2>{f(z)=}\alert<3>{\textbackslash frac1\{2\textbackslash pi i\}}\par
~~~~\alert<4>{\textbackslash oint\textbackslash limits\_%
\{\textbackslash partial U\_\textbackslash varepsilon(z)\}}\par
~~~~\alert<5>{\textbackslash frac\{f(\textbackslash
zeta)\textbackslash,d\textbackslash zeta\}\{\textbackslash zeta-z\}}\$\par
\end{programlisting}
\end{center}
\end{frame}

\begin{frame}{Где почитать про \TeX}
Конечно, всех этих примеров недостаточно, чтобы освоить все \TeX{}нические
тонкости.

Для более подробного знакомства с~системой \TeX\ предлагаем другие источники:

\begin{thebibliography}{}
\bibitem{bib:Knuth:TheTeXbookRDTeX}
Всё про \TeX
\newblock
Кнут, Д.~Е.
\newblock
Протвино~: АО RD\TeX, 1993.~— ISBN~5-900614-01-8

\bibitem{bib:Knuth:TheTeXbookWilliams}
Всё про \TeX
\newblock
Кнут, Дональд~Э.
\newblock
Москва~: Вильямс, 2003.~— ISBN~5-8459-0382-3~(рус.)
\end{thebibliography}
\end{frame}

\begin{frame}{Где почитать про \TeX\ {\mdseries\itshape(продолжение)}}
\begin{thebibliography}{}
\bibitem{bib:Lvovskij:LaTeX}
Набор и~вёрстка в~системе \LaTeX
\newblock
Львовский, С. М.
%— 3-е изд., испр. и доп.
\newblock
Москва~: МЦНМО, 2003.~— ISBN~5-94057-091-7
\end{thebibliography}
\end{frame}

\begin{frame}{\TeX{}овские фрагменты как объекты \literal{picture}}
\TeX{}овские вставки с~точки зрения \hologo{METAPOST} являются объектами типа
\literal{picture}. Код на~языке \TeX\ следует поместить между словами
\literal{btex} и~\literal{etex}:
\begin{center}
\Large
\literal{btex~}\replaceable{\TeX{}овский код}\literal{~etex}
\end{center}

Например:
\begin{programlisting}
btex \$\textbackslash frac1\{\textbackslash sqrt3\}\$ etex
\end{programlisting}
\end{frame}

\begin{frame}{Препроцессор}
Перед тем как \hologo{METAPOST} приступает к~компиляции исходного текста, этот
текст подвергается обработке препроцессором.

Препроцессор отыскивает фрагменты \literal{btex~…~etex}, извлекает их
в~\TeX{}овский файл, затем файл компилируется \TeX{}ом. Результат этой
компиляции обрабатывается специальной программой, преобразующей его в~код на
языке \hologo{METAPOST}. Этот код вставляется в~исходный текст рисунка как
выражение типа \literal{picture}. Компилятор \hologo{METAPOST} видит только
лишь результат этой закулисной деятельности.
\end{frame}

\begin{frame}{\TeX{}овские ошибки}
Нет необходимости вникать в~эти подробности, если \TeX{}овский код в~порядке.
Если~же он содержит ошибки с~точки зрения \TeX, \hologo{METAPOST} выдаёт ошибку
\begin{screen}
\only<1>{! Unable to make mpx file.}%
\only<2>{! Невозможно создать mpx-файл.}%
\end{screen}
Для исправления таких ошибок необходимы определённые \TeX{}нические познания.
\transdissolve<2>
\end{frame}

\begin{frame}{Вставка \TeX{}овских фрагментов в~\hologo{METAPOST}}
Если картинку \literal{btex~…~etex} нужно включить в~рисунок, можно
использовать команду \literal{draw}:
\begin{programlisting}
draw btex \$\textbackslash frac1\{\textbackslash sqrt3\}\$ etex;
\end{programlisting}

Однако такая команда размещает картинку таким образом, что левый нижний угол
картинки попадает в~начало координат. Для более точного размещения картинку
нужно сдвинуть.
\end{frame}

\begin{frame}{Команда \literal{label}}
В~библиотеке \nolinkurl{plain.mp} определена макрокоманда \literal{label},
которая позволяет разместить \TeX{}овскую надпись либо по~центру в~данной точке
рисунка, либо поблизости в~стороне от~данной точки. Поддерживаются восемь
положений: справа, сверху, слева и~снизу, и~ещё четыре промежуточных.

Синтаксис команды:
\begin{center}
\Large
\literal{label(}\replaceable{картинка}\literal{,~}\replaceable{точка}\literal{)}
\literal{label.}\replaceable{суффикс}\literal{(}\replaceable{картинка}\literal{, }\replaceable{точка}\literal{)}
\end{center}
\end{frame}

\begin{frame}{Команда \literal{label.}\replaceable{суффикс}}
В~команде \literal{label} за~положение \replaceable{картинки} по~отношению
к~\replaceable{точке} отвечает \replaceable{суффикс}. Допустимые
\replaceable{суффиксы}:

\bigskip

\centerline{\includeMPgraphics{figure-label-suffices}}
\end{frame}

\begin{frame}{Пример размещения надписей}
\begin{columns}
\column{.33333\textwidth}
\begin{center}
\only<1>{\includeMPgraphics{figure-octagon}}%
\only<2>{\includeMPgraphics{figure-octagon-label-rt}}%
\only<3>{\includeMPgraphics{figure-octagon-label-urt}}%
\only<4>{\includeMPgraphics{figure-octagon-label-top}}%
\only<5>{\includeMPgraphics{figure-octagon-label-ulft}}%
\only<6>{\includeMPgraphics{figure-octagon-label-lft}}%
\only<7>{\includeMPgraphics{figure-octagon-label-llft}}%
\only<8>{\includeMPgraphics{figure-octagon-label-bot}}%
\only<9>{\includeMPgraphics{figure-octagon-label-lrt}}%
\only<10>{\includeMPgraphics{figure-octagon-label}}%
\end{center}
\column{.66667\textwidth}
\only<1>{%
\large
\begin{programlisting}
draw\par
~~~~for d:=0 upto 7:\par
~~~~~~~~cm*dir(45d)--\par
~~~~endfor\par
~~~~cycle;\par
\end{programlisting}%
}%
\only<2-10>{%
\large
\begin{programlisting}
label\only<2-9>{.}%
\alert<2-9>{%
\only<2>{rt}%
\only<3>{urt}%
\only<4>{top}%
\only<5>{ulft}%
\only<6>{lft}%
\only<7>{llft}%
\only<8>{bot}%
\only<9>{lrt}%
}%
\only<10>{}%
(\par
~~~~~~~~btex \$\alert<2-10>{P%
\only<2>{\_0}%
\only<3>{\_1}%
\only<4>{\_2}%
\only<5>{\_3}%
\only<6>{\_4}%
\only<7>{\_5}%
\only<8>{\_6}%
\only<9>{\_7}%
\only<10>{}%
}\$ etex,\par
~~~~~~~~%
\alert<2-10>{%
\only<2>{cm*right}%
\only<3>{cm*dir 45}%
\only<4>{cm*up}%
\only<5>{cm*dir 135}%
\only<6>{cm*left}%
\only<7>{cm*dir 225}%
\only<8>{cm*down}%
\only<9>{cm*dir 315}%
\only<10>{origin}%
}\par
~~~~);
\end{programlisting}%
}%
\end{columns}
\transdissolve<2-10>
\end{frame}
