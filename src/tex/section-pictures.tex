\sectionwithlogo{Картинки}
	{\includeMPgraphics{figure-houses}}

\begin{frame}{Переменная \literal{currentpicture}}
В~процессе рисования с~помощью команд \literal{draw} и~\literal{fill} текущее
изображение сохраняется в~переменной \literal{currentpicture} типа
\literal{picture}.

Команда \literal{endfig}, помимо прочих действий, выводит содержимое переменной
\literal{currentpicture} в~файл в~виде последовательности команд на языке
PostScript. Команда \literal{beginfig} «обнуляет» эту переменную, а~также
присваивает номер будущей картинки (обычно этот номер включается в~имя
PostScript-файла).

Внутренняя структура переменных типа \literal{picture} сложна. Она позволяет
хранить элементы изображения (пути) вместе с~атрибутами рисования (цвет линии,
её толщина, цвет заливки циклического контура), а~также последовательность,
в~которой эти элементы рисовались.
\end{frame}

\begin{frame}{Рисование на картинках и~команда \literal{addto}}
Имеется возможность заготавливать картинки впрок для многократного
использования. Можно создавать библиотеки готовых картинок.

Для рисования на картинках, отличных от \literal{currentpicture},
в~\hologo{METAPOST} имеется команда \literal{addto}:
\begin{center}
\only<1>{{\large\literal{addto~}\replaceable{картинка}\literal{~doublepath~}\replaceable{путь}~\replaceable{опции}}\\[3ex]
(аналог \literal{draw~}\replaceable{путь}~\replaceable{опции})}%
\only<2>{{\large\literal{addto~}\replaceable{картинка}\literal{~contour~}\replaceable{циклический
путь}~\replaceable{опции}}\\[3ex]
(аналог \literal{fill~}\replaceable{циклический путь}~\replaceable{опции})}%
\only<3>{{\large\literal{addto~}\replaceable{картинка}\literal{~also~}\replaceable{картинка}}\\[3ex]
(аналог \literal{draw~}\replaceable{картинка})}%
\end{center}
\end{frame}

\begin{frame}{Пример: сумасшедшие домики}
\begin{columns}
\column{.5\textwidth}
Проиллюстрируем сказанное примером~— рисованием посёлка из~сумасшедших домиков
(это рисунок, открывающий настоящий раздел).

\column{.5\textwidth}
\centerline{\includeMPgraphics{figure-houses}}
\end{columns}
\end{frame}

\begin{frame}{Пример: сумасшедшие домики}
Все домики на~рисунке отличаются друг от~друга разными аффинными
преобразованиями, выбранными случайно. Они равномерно разбросаны по~полю
рисунка, имеют размеры, немного и~случайно отклоняющиеся от~некоторого среднего
размера, а~также небольшие и~случайные наклоны.

\bigskip

\begin{columns}[onlytextwidth]
\column{.75\textwidth}
Все эти сумасшедшие домики получаются нанесением на поле рисунка одной и~той~же
картинки (на~ней изображён один совершенно нормальный домик), подвергнутой
различным аффинным преобразованиям. Картинка заранее заготовлена в~переменной
\literal{house}, имеющей тип \literal{picture}.
\column{.25\textwidth}
\centerline{\includeMPgraphics{figure-house}}
\end{columns}
\end{frame}

\begin{frame}{Пример: сумасшедшие домики}
\only<1>{Объявляем новую переменную-картинку \literal{house}.
Инициализируем её пустой картинкой…}%
\only<2>{Рисуем в~переменную \literal{house} крышу…}%
\only<3>{Стену…}%
\only<4>{Чёрный проём окна…}%
\only<5>{Оконную раму…}%
\only<6>{И, наконец, наносим на наш рисунок (то~есть в~переменную
\literal{currentpicture}) 50~копий картинки \literal{house}, подвергнутых
сумасшедшим аффинным преобразованиям. Пояснения последуют.}%
\begin{programlisting}
\only<1>{%
picture house;\par
house=nullpicture;\par}%
\only<2>{%
addto house contour\par
~~~~(-.2, 1)--(1.2, 1)--(.5, 1.5)--cycle\par
~~~~withcolor .5orange;\par}%
\only<3>{%
addto house contour unitsquare\par
~~~~withcolor .5red;\par}%
\only<4>{%
addto house contour\par
~~~~unitsquare scaled .5 shifted (.25, .25)\par
~~~~withcolor black;\par}%
\only<5>{%
addto house doublepath\par
~~~~unitsquare scaled .5 shifted (.25, .25)\par
~~~~withcolor orange;\par
addto house doublepath\par
~~~~(.25, .5)--(.75, .5)\par
~~~~withcolor orange;\par
addto house doublepath\par
~~~~(.5, .25)--(.5, .75)\par
~~~~withcolor orange;\par}%
\only<6>{%
for i:=1 upto 50:\par
~~~~draw house\par
~~~~~~~~slanted (normaldeviate*.005cm)\par
~~~~~~~~scaled (.25cm+normaldeviate*.05cm)\par
~~~~~~~~shifted (uniformdeviate(3cm), uniformdeviate(4cm));\par
endfor}%
\end{programlisting}
\end{frame}

\begin{frame}{Команда \literal{uniformdeviate}}
\begin{columns}[onlytextwidth]
\column{.25\textwidth}
\includeMPgraphics{figure-uniformdeviate}
\column{.75\textwidth}

По площади квадратика размером 2~см равномерно разбросаны 300~точек, чьи
координаты (и~абсцисса, и~ордината) получены с~помощью команды
\literal{uniformdeviate(2cm)}.

\end{columns}

Примитивная команда
\literal{uniformdeviate(}\replaceable{число}\literal{)}
возвращает (псевдо)случайное число, равномерно распределённое на~промежутке
от~$0$ до~\replaceable{числа}.

\begin{programlisting}
for i:=1 upto 300:\par
~~~~draw \alert{(uniformdeviate(2cm), uniformdeviate(2cm))}\par
~~~~~~~~withpen pensquare scaled bp\par
~~~~~~~~withcolor red;\par
endfor
\end{programlisting}
\end{frame}

\begin{frame}{Команда \literal{normaldeviate}}
\begin{columns}[onlytextwidth]
\column{.25\textwidth}
\includeMPgraphics{figure-normaldeviate}
\column{.75\textwidth}
А~теперь разбросаны 300~точек, чьи координаты получены с~помощью команды
\literal{normaldeviate}. Это похоже на~прицельную стрельбу в~центр квадрата,
при которой кучность по~горизонтали несколько выше, чем по~вертикали.
\end{columns}
\begin{programlisting}
for i:=1 upto 300:\par
~~~~draw \alert{(cm+normaldeviate*2mm, cm+normaldeviate*5mm)}\par
~~~~~~~~withpen pensquare scaled bp\par
~~~~~~~~withcolor red;\par
endfor
\end{programlisting}
\end{frame}

\begin{frame}{Команда \literal{normaldeviate}}
Примитивная команда \literal{normaldeviate} возвращает
\emph{нормально"=распределённое} (псевдо)случайное число. Плотность
распределения выше для чисел вблизи нуля и~убывает до~нуля на~отдалении.
Бо́льшие по~модулю числа имеют меньшую вероятность появления. Эта плотность
описывается формулой $e^{-x^2\fracslash2}$, её график имеет колоколообразную
форму:
\begin{center}
\includeMPgraphics{figure-gaussdensity}
\end{center}

Такое распределение ещё называют \emph{гауссовским}.
\end{frame}

\begin{frame}{Вырезание}
\begin{columns}[onlytextwidth,c]
\column{.33333\textwidth}
\only<1>{\includeMPgraphics{figure-clipping-1}}%
\only<2>{\includeMPgraphics{figure-clipping-2}}%
\column{.6\textwidth}
\begin{programlisting}
for j:=-10 upto 10:\par
~~~~for i:=-10 upto 10:\par
~~~~~~~~fill fullcircle\par
~~~~~~~~~~~~scaled mm\par
~~~~~~~~~~~~shifted ((i, j)*1.5mm)\par
~~~~~~~~~~~~withcolor .5red;\par
~~~~endfor\par
endfor\par
\uncover<2>{%
\leavevmode\par
\alert{clip} currentpicture\par
~~~~\alert{to} fullcircle scaled 3cm;\par}%
\end{programlisting}
\end{columns}
\end{frame}

\begin{frame}{Текстуры}
\begin{columns}[onlytextwidth]
\column{.33333\textwidth}
\centering
\only<1>{\includeMPgraphics{figure-texture-1}}%
\only<2>{\includeMPgraphics{figure-texture-2}}%
\only<3>{\includeMPgraphics{figure-texture-3}}%
\column{.6\textwidth}
\only<1-2>{

Если требуется наложить описанное изображение на рисунок, где уже что-то
изображено, описанный приём не сработает.

Дело в~том, что последняя команда \literal{clip currentpicture to~…} ограничит
всё ранее нарисованное внутри пути \literal{fullcircle scaled 3cm}.}%
\only<3>{%
Выход из~положения заключается в~том, чтобы наносить текстуру не прямо
на~текущий рисунок в~переменной \literal{currentpicture}, а~на~вспомогательную
картинку. Затем нужно применить команду \literal{clip} к~этой вспомогательной
картинке. Наконец, следует наложить вспомогательную картинку
на~\literal{currentpicture}.}
\end{columns}
\end{frame}

\begin{frame}{Текстуры: код\only<2>{ {\mdseries\itshape(продолжение)}}}
\begin{columns}[onlytextwidth]
\column{.33333\textwidth}
\centering
\includeMPgraphics{figure-texture-3}%
\column{.6\textwidth}
\begin{programlisting}
\only<1>{fill unitsquare scaled 2.5cm\par%
~~~~shifted (-1.25cm, 0)\par
~~~~withcolor .5green;\par
\leavevmode\par
\alert{picture tmpPic;\par
tmpPic:=nullpicture;\par}%
\leavevmode\par
for j:=-10 upto 10:\par
~~~~for i:=-10 upto 10:\par
~~~~~~~~\alert{addto tmpPic contour}\par
~~~~~~~~~~~~fullcircle scaled mm\par
~~~~~~~~~~~~shifted ((i, j)*1.5mm)\par
~~~~~~~~~~~~withcolor .5red;\par
~~~~endfor\par
endfor\par}%
\only<2>{clip \alert{tmpPic} to fullcircle\par
~~~~scaled 3cm;\par
\leavevmode\par
\alert{addto currentpicture also tmpPic;}}
\end{programlisting}
\end{columns}
\end{frame}
