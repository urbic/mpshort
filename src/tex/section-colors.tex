\sectionwithlogo{Цвета}{\scalebox{.85}{\includeMPgraphics{figure-colorcircles}}}

%%%%%%%%%%

\begin{frame}{Цветовая схема~RGB}
\begin{columns}
\column{.5\textwidth}
\includeMPgraphics{figure-colorcircles}
\column{.5\textwidth}
В~\hologo{METAPOST} возможно цветное рисование. Для представления цветов
предусмотрен тип \literal{color}.

\bigskip

В~\hologo{METAPOST} принята трёхкомпонентная цветовая схема~RGB. Каждый цвет
представляется тройкой чисел~$(R,G,B)$, каждое из~которых заключено
в~отрезке~$[0;1]$.
\end{columns}

Каждый цвет в~модели~RGB считается смесью трёх «основных» цветов. Числа~$R$,
$G$, $B$~— количества соответственно красного, зелёного и~синего цветов в~такой
смеси.
\end{frame}

%%%%%%%%%%

\begin{frame}{Цветовой треугольник}
\centerline{\includeMPgraphics{figure-colortriangle}}
\end{frame}

%%%%%%%%%%

\begin{frame}{Предопределённые цветовые переменные}
\begin{columns}
\column{.33333\textwidth}
\rightline{\includeMPgraphics{figure-square-red}}%
\column{.66667\textwidth}
\leftline{\LARGE\literal{red}\enspace→\enspace\literal{(1, 0, 0)}}%
\end{columns}
\bigskip
\begin{columns}
\column{.33333\textwidth}
\rightline{\includeMPgraphics{figure-square-green}}%
\column{.66667\textwidth}
\leftline{\LARGE\literal{green}\enspace→\enspace\literal{(0, 1, 0)}}%
\end{columns}
\bigskip
\begin{columns}
\column{.33333\textwidth}
\rightline{\includeMPgraphics{figure-square-blue}}%
\column{.66667\textwidth}
\leftline{\LARGE\literal{blue}\enspace→\enspace\literal{(0, 0, 1)}}%
\end{columns}
\end{frame}

%%%%%%%%%%

\begin{frame}{Цветовые выражения}
\only<1>{%
\begin{columns}
\column{.5\textwidth}
\rightline{\includeMPgraphics{figure-square-darkred}}%
\column{.5\textwidth}
\leftline{\LARGE\literal{.7red}}%
\end{columns}
\bigskip
\begin{columns}
\column{.5\textwidth}
\rightline{\includeMPgraphics{figure-square-darkgreen}}%
\column{.5\textwidth}
\leftline{\LARGE\literal{.7green}}%
\end{columns}
\bigskip
\begin{columns}
\column{.5\textwidth}
\rightline{\includeMPgraphics{figure-square-darkblue}}%
\column{.5\textwidth}
\leftline{\LARGE\literal{.7blue}}%
\end{columns}
}%
\only<2>{%
\begin{columns}
\column{.5\textwidth}
\rightline{\includeMPgraphics{figure-square-red+green}}%
\column{.5\textwidth}
\leftline{\LARGE\literal{red+green}}%
\end{columns}
\bigskip
\begin{columns}
\column{.5\textwidth}
\rightline{\includeMPgraphics{figure-square-red,green}}%
\column{.5\textwidth}
\leftline{\LARGE\literal{.5[red, green]}}%
\end{columns}
}%
\end{frame}

%%%%%%%%%%

\begin{frame}{Определение цветовых переменных}
\begin{programlisting}
color yellow, cyan, magenta;\par
\leavevmode\par
yellow:=\only<1>{red+green}\only<2>{(1, 1, 0)};\par
cyan:=\only<1>{green+blue}\only<2>{(0, 1, 1)};\par
magenta:=\only<1>{blue+red}\only<2>{(1, 0, 1)};
\end{programlisting}
\end{frame}

%%%%%%%%%%

\begin{frame}{Декомпозиция цветов}
Операторы \literal{redpart}, \literal{greenpart} и~\literal{bluepart} извлекают
из цветового выражения соответственно красную, зелёную и~синюю составляющую:
\begin{programlisting}
color tomato;\par
tomato=(1, .388235294117647, .27843137254902);
\end{programlisting}

\bigskip

\begin{columns}
\column{.33333\textwidth}
\includeMPgraphics{figure-colorcircles-tomato}
\column{.66667\textwidth}
\begin{center}
\large
\begin{grid}{r@{\enspace→\enspace}l}
\literal{redpart tomato}&\literal{1}\\
\literal{greenpart tomato}&\literal{.38823}\\
\literal{bluepart tomato}&\literal{.27843}
\end{grid}
\end{center}
\end{columns}
\end{frame}
