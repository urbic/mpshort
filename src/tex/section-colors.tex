\sectionwithlogo{Цвета}{\scalebox{.85}{\leavevmode
\begin{mplibcode}
beginfig(0)
u:=cm;

path c[];
c1=fullcircle scaled 3u shifted (.75u*sqrt 3*dir -150);
c2=fullcircle scaled 3u shifted (.75u*sqrt 3*dir -30);
c3=fullcircle scaled 3u shifted (.75u*sqrt 3*dir 90);
c4=buildcycle(c1 rotatedabout((.75u*sqrt 3*dir -150), 90), c2);
c5=buildcycle(c2, c3);
c6=buildcycle(c3, c1);
c7=buildcycle(c1, c2, c3);

fill c1 withcolor red;
fill c2 withcolor green;
fill c3 withcolor blue;
fill c4 withcolor red+green;
fill c5 withcolor green+blue;
fill c6 withcolor blue+red;
fill c7 withcolor white;

endfig
\end{mplibcode}
}}

\begin{frame}{Цветовая схема~RGB}
\begin{columns}
\column{.5\textwidth}
\leavevmode
\begin{mplibcode}
beginfig(0)
u:=cm;

path c[];
c1=fullcircle scaled 3u shifted (.75u*sqrt 3*dir -150);
c2=fullcircle scaled 3u shifted (.75u*sqrt 3*dir -30);
c3=fullcircle scaled 3u shifted (.75u*sqrt 3*dir 90);
c4=buildcycle(c1 rotatedabout((.75u*sqrt 3*dir -150), 90), c2);
c5=buildcycle(c2, c3);
c6=buildcycle(c3, c1);
c7=buildcycle(c1, c2, c3);

fill c1 withcolor red;
fill c2 withcolor green;
fill c3 withcolor blue;
fill c4 withcolor red+green;
fill c5 withcolor green+blue;
fill c6 withcolor blue+red;
fill c7 withcolor white;

endfig
\end{mplibcode}

\column{.5\textwidth}
В~\hologo{METAPOST} возможно цветное рисование. Для представления цветов
предусмотрен тип \literal{color}.

\bigskip

В~\hologo{METAPOST} принята трёхкомпонентная цветовая схема~RGB. Каждый цвет
представляется тройкой чисел~$(R,G,B)$, каждое из~которых заключено
в~отрезке~$[0;1]$.
\end{columns}

Каждый цвет в~модели~RGB считается смесью трёх «основных» цветов. Числа~$R$,
$G$, $B$~— количества соответственно красного, зелёного и~синего цветов в~такой
смеси.
\end{frame}

\begin{frame}{Цветовой треугольник}
\centerline{\leavevmode
\begin{mplibcode}
beginfig(0)
u:=cm;

z1=origin;
z2=(5u, 0);
z3=.5[z1, z2]+(sqrt 3)/2*((z2-z1) zscaled up);
numeric k;
k=32;

%draw z1--z2--z3--cycle;

for i:=0 upto k-1:
	for j:=0 upto k-i-1:
		filldraw (z1--z2--z3--cycle) scaled (1/k) shifted (i/k*(z2-z1)+j/k*(z3-z1))
			withcolor (j/k)[(i/k)[red, green], blue];
		if j<k-i-1:
			filldraw (z1--z2--z3--cycle) scaled (-1/k) shifted ((i+1)/k*(z2-z1)+(j+1)/k*(z3-z1))
				withcolor ((j+.5)/k)[((i+.5)/k)[red, green], blue];
		fi
	endfor
endfor

label.llft("\large\textbf R", z1) withcolor red;
label.lrt("\large\textbf G", z2) withcolor green;
label.top("\large\textbf B", z3) withcolor blue;

endfig
\end{mplibcode}
}
\end{frame}

\begin{frame}{Предопределённые цветовые переменные}
\begin{columns}
\column{.33333\textwidth}
\rightline{\leavevmode
\begin{mplibcode}
beginfig(0)

fill unitsquare scaled 2cm withcolor red;

endfig
\end{mplibcode}
}%
\column{.66667\textwidth}
\leftline{\LARGE\literal{red}${}\equiv{}$\literal{(1, 0, 0)}}%
\end{columns}
\bigskip
\begin{columns}
\column{.33333\textwidth}
\rightline{\leavevmode
\begin{mplibcode}
beginfig(0)

fill unitsquare scaled 2cm withcolor green;

endfig
\end{mplibcode}
}%
\column{.66667\textwidth}
\leftline{\LARGE\literal{green}${}\equiv{}$\literal{(0, 1, 0)}}%
\end{columns}
\bigskip
\begin{columns}
\column{.33333\textwidth}
\rightline{\leavevmode
\begin{mplibcode}
beginfig(0)

fill unitsquare scaled 2cm withcolor blue;

endfig
\end{mplibcode}
}%
\column{.66667\textwidth}
\leftline{\LARGE\literal{blue}${}\equiv{}$\literal{(0, 0, 1)}}%
\end{columns}
\end{frame}

\begin{frame}{Цветовые выражения}
\only<1>{%
\begin{columns}
\column{.5\textwidth}
\rightline{\leavevmode
\begin{mplibcode}
beginfig(0)

fill unitsquare scaled 2cm withcolor .7red;

endfig
\end{mplibcode}
}%
\column{.5\textwidth}
\leftline{\LARGE\literal{.7red}}%
\end{columns}
\bigskip
\begin{columns}
\column{.5\textwidth}
\rightline{\leavevmode
\begin{mplibcode}
beginfig(0)

fill unitsquare scaled 2cm withcolor .7green;

endfig
\end{mplibcode}
}%
\column{.5\textwidth}
\leftline{\LARGE\literal{.7green}}%
\end{columns}
\bigskip
\begin{columns}
\column{.5\textwidth}
\rightline{\leavevmode
\begin{mplibcode}
beginfig(0)

fill unitsquare scaled 2cm withcolor .7blue;

endfig
\end{mplibcode}
}%
\column{.5\textwidth}
\leftline{\LARGE\literal{.7blue}}%
\end{columns}
}%
\only<2>{%
\begin{columns}
\column{.5\textwidth}
\rightline{\leavevmode
\begin{mplibcode}
beginfig(0)

fill unitsquare scaled 2cm withcolor red+green;

endfig
\end{mplibcode}
}%
\column{.5\textwidth}
\leftline{\LARGE\literal{red+green}}%
\end{columns}
\bigskip
\begin{columns}
\column{.5\textwidth}
\rightline{\leavevmode
\begin{mplibcode}
beginfig(0)

fill unitsquare scaled 2cm withcolor .5[red, green];

endfig
\end{mplibcode}
}%
\column{.5\textwidth}
\leftline{\LARGE\literal{.5[red, green]}}%
\end{columns}
}%
\end{frame}

\begin{frame}{Определение цветовых переменных}
\begin{programlisting}
color yellow, cyan, magenta;\par
\leavevmode\par
yellow:=\only<1>{red+green}\only<2>{(1, 1, 0)};\par
cyan:=\only<1>{green+blue}\only<2>{(0, 1, 1)};\par
magenta:=\only<1>{blue+red}\only<2>{(1, 0, 1)};
\end{programlisting}
\end{frame}

\begin{frame}{Декомпозиция цветов}
Операторы \literal{redpart}, \literal{greenpart} и~\literal{bluepart} извлекают
из цветового выражения соответственно красную, зелёную и~синюю составляющую:
\begin{programlisting}
color tomato;\par
tomato=(1, .388235294117647, .27843137254902);
\end{programlisting}

\bigskip

\begin{columns}
\column{.25\textwidth}
\leavevmode
\begin{mplibcode}
beginfig(0)

color tomato;
tomato=(1, 0.388235294117647, 0.27843137254902);
path c[];
c1=fullcircle scaled 2cm shifted (.5cm*sqrt 3*dir -150);
c2=fullcircle scaled 2cm shifted (.5cm*sqrt 3*dir -30);
c3=fullcircle scaled 2cm shifted (.5cm*sqrt 3*dir 90);
c4=buildcycle(c1 rotatedabout((.5cm*sqrt 3*dir -150), 90), c2);
c5=buildcycle(c2, c3);
c6=buildcycle(c3, c1);
c7=buildcycle(c1, c2, c3);

fill c1 withcolor (redpart tomato, 0, 0);
fill c2 withcolor (0, greenpart tomato, 0);
fill c3 withcolor (0, 0, bluepart tomato);
fill c4 withcolor (redpart tomato, greenpart tomato, 0);
fill c5 withcolor (0, greenpart tomato, bluepart tomato);
fill c6 withcolor (redpart tomato, 0, bluepart tomato);
fill c7 withcolor tomato;

endfig
\end{mplibcode}

\column{.75\textwidth}
\begin{center}
\large
\begin{grid}{rcl}
\literal{redpart tomato}%
&${}\equiv{}$
&\literal{1}
\\
\literal{greenpart tomato}%
&${}\equiv{}$
&\literal{.38823}%
\\
\literal{bluepart tomato}%
&${}\equiv{}$
&\literal{.27843}%
\end{grid}
\end{center}
\end{columns}
\end{frame}
