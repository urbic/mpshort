\sectionwithlogo{Уравнения}
	{\leavevmode
\begin{mplibcode}
beginfig(0)

z1=origin;
z2=(3cm, 1.5cm);
z3=(2cm, -cm);
z4=altbase(z3, z1, z2);

draw z1--z2;
draw z3--z4 withpen thinpen dashed evenly scaled .5;

markdot.circle(z1) fg=>orange;
markdot.circle(z2) fg=>orange;
markdot.circle(z3) fg=>orange;
markdot.circle(z4) fg=>orange;

% TODO markangle.rt(z3, z4, z2, 1) fg=>orange, extra=>3;

label.top("$z_1$", z1);
label.top("$z_2$", z2);
label.bot("$z_3$", z3);
label.top("$z$", z4);

endfig
\end{mplibcode}
}

\begin{frame}{Уравнения в~\hologo{METAPOST}}
В~\hologo{METAPOST} имеется уникальный встроенный механизм решения линейных
алгебраических уравнений.

Наиболее распространённое применение этого механизма~— вычисление точки
пересечения двух прямых, каждая из которых задана парой точек.
\end{frame}

\begin{frame}{Пересечение двух прямых}
\begin{columns}
\column{.3\textwidth}
\leavevmode
\begin{mplibcode}
beginfig(0)

z1=origin;
z2=(2cm, 5cm);
z3=(2.5cm, cm);
z4=(.5cm, 4cm);
z5=intersection(z1, z2)(z3, z4);

draw z1--z2;
draw z3--z4;

markdot.circle(z1) fg=>orange;
markdot.circle(z2) fg=>orange;
markdot.circle(z3) fg=>orange;
markdot.circle(z4) fg=>orange;
markdot.circle(z5) fg=>orange;

label.bot("$z_1$", z1);
label.bot("$z_3$", z3);
label.top("$z_2$", z2);
label.top("$z_4$", z4);
label.rt("$z$", z5+.1cm*right);

endfig
\end{mplibcode}

\column{.7\textwidth}
Утверждение о~том, что точка~$z$ лежит на прямой~$z_1z_2$, алгебраически
выражается так:
	\[
	z=\only<1>{\lambda[z_1,z_2]}\only<2>{(1-\lambda)z_1+\lambda z_2},
	\tag{1}
	\]
где~$\lambda$~— некоторое число.
Аналогично, если точка~$z$ к~тому~же лежит на другой прямой~$z_3z_4$, получим
	\[
	z=\only<1>{\mu[z_3,z_4]}\only<2>{(1-\mu)z_3+\mu z_4}.
	\tag{2}
	\]
Достаточно найти~$\lambda$, приравнивая правые части уравнений~\thetag1,
\thetag2, и~подставить в~\thetag1, чтобы найти выражение для~$z$.
\end{columns}
\end{frame}

\begin{frame}{Пересечение двух прямых}
Векторное уравнение
	\[
	(1-\lambda)z_1+\lambda z_2=(1-\mu)z_3+\mu z_4
	\]
можно рассматривать как систему из двух скалярных уравнений с~двумя
неизвестными~$\lambda$, $\mu$:
	\[
	\left\{
	\begin{aligned}
		&(1-\lambda)x_1+\lambda x_2=(1-\mu)x_3+\mu x_4,\\
		&(1-\lambda)y_1+\lambda y_2=(1-\mu)y_3+\mu y_4.
	\end{aligned}
	\right.
	\]
Выражение для~$\lambda$ получается довольно громоздким:
	\[
	\lambda=\frac{(x_3-x_1)(y_4-y_3)-(x_4-x_3)(y_3-y_1)}
		{(x_2-x_1)(y_4-y_3)-(x_4-x_3)(y_2-y_1)},
	\]
а~для~$z$~— и~того хуже.
\end{frame}

\begin{frame}{Пересечение двух прямых}
Встроенный в~\hologo{METAPOST} механизм решения линейных алгебраических
уравнений избавляет нас от всех этих вычислений:
\begin{programlisting}
numeric lambda, mu;\par
pair z;\par
z=lambda[z1, z2]=mu[z3, z4];
\end{programlisting}
Теперь пара~\literal z получит нужное значение.
\end{frame}

\begin{frame}{Команда \literal{whatever}}
Обратите внимание на то, что из трёх найденных значений~$z$, $\lambda$, $\mu$
нас в~конечном итоге интересовало только~$z$.

Переменные $\lambda$ и~$\mu$ являются вспомогательными, они отвечают за
положение точки~$z$ на прямых~$z_1z_2$ и~$z_3z_4$.

Чтобы не изобретать имена для вспомогательных числовых переменных, чтобы не
объявлять эти переменные, удобно воспользоваться командой \literal{whatever}:
\begin{programlisting}
pair z;\par
z=whatever[z1, z2]=whatever[z3, z4];
\end{programlisting}
\end{frame}

\begin{frame}{Команда \literal{whatever}}
Команда \literal{whatever} создаёт новую безымянную числовую переменную
и~тут~же её подставляет в~текст программы.

Название этой команды очень хорошо отражает её предназначение: выражение
\literal{z=whatever[z1, z2]} значит «точка~$z$ лежит где"=то
на~прямой~$z_1z_2$».

Обратите внимание: каждая команда~\literal{whatever} в~программе создаёт
числовую переменную, полностью независимую от других, созданных той~же
командой.
\end{frame}

\begin{frame}{Пример: проекция точки на~прямую}
\begin{columns}
\column{.35\textwidth}
\centering
\only<1>{\leavevmode
\begin{mplibcode}
beginfig(0)

z1=origin;
z2=(3cm, 1.5cm);
z3=(2cm, -cm);
z4=altbase(z3, z1, z2);

draw z1--z2;
draw z3--z4 withpen thinpen dashed evenly scaled .5;

markdot.circle(z1) fg=>orange;
markdot.circle(z2) fg=>orange;
markdot.circle(z3) fg=>orange;
markdot.circle(z4) fg=>orange;

% TODO markangle.rt(z3, z4, z2, 1) fg=>orange, extra=>3;

label.top("$z_1$", z1);
label.top("$z_2$", z2);
label.bot("$z_3$", z3);
label.top("$z$", z4);

endfig
\end{mplibcode}
}%
\only<2>{\leavevmode
\begin{mplibcode}
beginfig(0)

z1=origin;
z2=(3cm, 1.5cm);
z3=(2cm, -cm);
z4=altbase(z3, z1, z2);

draw z1--z2;
draw z3--z4 withpen thinpen dashed evenly scaled .5;

markdot.circle(z1) fg=>orange;
markdot.circle(z2) fg=>orange;
markdot.circle(z3) fg=>orange;
markdot.circle(z4) fg=>orange;

% TODO markangle.rt(z3, z4, z2, 1) fg=>orange, extra=>3;

label.top("$z_1$", z1);
label.top("$z_2$", z2);
label.bot("$z_3$", z3);
label.top("$z$", z4);

endfig
\end{mplibcode}
}%
\column{.65\textwidth}
Применим механизм решения уравнений для~вычисления ортогональной проекции точки
на~прямую.

\pause
\bigskip
Во"=первых, точка~$z$ лежит на~прямой~$z_1z_2$:
\begin{programlisting}
z=whatever[z1, z2];
\end{programlisting}
\end{columns}
\end{frame}

\begin{frame}{Пример: проекция точки на~прямую}
\begin{columns}
\column{.35\textwidth}
\centering
\leavevmode
\begin{mplibcode}
beginfig(0)

z1=origin;
z2=(3cm, 1.5cm);
z3=(2cm, -cm);
z4=altbase(z3, z1, z2);
z5=z3+((z2-z1) rotated 90);

drawarrow z1-- .975[z1, z2];
drawarrow z3-- .975[z3, z5]
	withpen thinpen
	dashed evenly scaled .5
	withcolor red;

markdot.circle(z1) fg=>orange;
markdot.circle(z2) fg=>orange;
markdot.circle(z3) fg=>red;
markdot.circle(z4) fg=>red;
markdot.circle(z5) fg=>red;

% TODO markangle.rt(z3, z4, z2, 1) fg=>orange, extra=>3;

label.bot("$z_1$", z1);
label.top("$z_2$", z2);
label.bot("$z_3$", z3);
label.top("$z$", z4);

endfig
\end{mplibcode}

\column{.65\textwidth}
Во"=вторых, $z$ находится на~перпендикуляре к~прямой~$z_1z_2$, проходящем через
точку~$z_3$.

\bigskip

Вторую недостающую точку этого перпендикуляра можно найти, сдвинув точку~$z_3$
на вектор~$z_1z_2$, повёрнутый на~$90^\circ$:
\begin{programlisting}
z=whatever[z3,\par
~~~~~~~~z3+((z2-z1) rotated 90)];
\end{programlisting}
\end{columns}
\end{frame}

\begin{frame}{Способы придания значений переменным}
В~\hologo{METAPOST} переменная может получить значение двумя способами:
\begin{columns}
\column{.5\textwidth}
\begin{itemize}
\item
с~помощью \alert<2>{присваивания}
\item
с~помощью \alert<3>{уравнений}
\end{itemize}
\column{.5\textwidth}
\begin{center}
\LARGE\literal{%
\only<2>{a\alert{:=}5}%
\only<3>{8\alert=a+3}%
}%
\end{center}
\end{columns}
\end{frame}

\begin{frame}{Синтаксис присваивания}
\begin{center}
\LARGE
\replaceable{переменная}\literal{:=}\replaceable{выражение}\literal{;}
\end{center}
\end{frame}

\begin{frame}{Синтаксис уравнения}
\only<1>{%
\begin{center}
\Large
\replaceable{линейное выражение}\literal{=}\replaceable{линейное выражение}\literal{;}
\end{center}}%
\only<2>{Правая и~левая части уравнения должны иметь одинаковый тип.

\replaceable{линейное выражение} должно быть линейным по отношению ко~всем
неизвестным числам.

Допускаются комбинации из нескольких уравнений:
\begin{programlisting}
z4-z1=z3-z2=(3, y5);
\end{programlisting}
}%
\end{frame}

\begin{frame}{Несовместные уравнения}
\only<1->{%
По мере появления в~программе уравнений значения участвующих в~них переменных
уточняются.
\begin{programlisting}
x+y=1;\par
3-2y=2x;
\end{programlisting}

}
\only<2-3>{%
При появлении уравнения, противоречащего предыдущим, возникает ошибка:
\begin{screen}
\only<2>{! Inconsistent equation (off by -1).}%
\only<3>{! Несовместное уравнение (отклонение на -1).}%
\end{screen}
}%
\transdissolve<3>
\end{frame}

\begin{frame}{Избыточные уравнения}
Уравнения, которые возможно преобразовать к~тривиальному уравнению
\literal{0=0}, вызывают ошибку:
\begin{programlisting}
3=5-2;
\end{programlisting}
\begin{screen}
\only<1>{! Redundant equation.}%
\only<2>{! Избыточное уравнение.}%
\end{screen}
\transdissolve<2>
\end{frame}
