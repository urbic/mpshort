\sectionwithlogo{Числа, пары, арифметика}{\includeMPgraphics{figure-sierpinski}}

%%%%%%%%%%

\begin{frame}{Числа и~пары}
В~\hologo{METAPOST} значения числового типа используются, помимо прочего, для
представления координат точек.

Сами точки представляются как пары координат.
\end{frame}

%%%%%%%%%%

\begin{frame}{Единицы масштаба}
Единицей масштаба служит большой типографский пункт (\literal{bp})~— ему
соответствует число \literal{1}. Для этой и~некоторых других традиционных для
полиграфии единиц измерения длин в~библиотеке \nolinkurl{plain.mp} определены
числовые переменные:

\begin{center}
\begin{tabular}{rrrl}
\small пункт&$0{,}99626$&\literal{pt}&\includeMPgraphics{figure-dimension-pt}\\
\small большой пункт&$1$&\literal{bp}&\includeMPgraphics{figure-dimension-bp}\\
\small пункт Дидо́&$1{,}06601$&\literal{dd}&\includeMPgraphics{figure-dimension-dd}\\
\small миллиметр&$2{,}83464$&\literal{mm}&\includeMPgraphics{figure-dimension-mm}\\
\small пика&$11{,}95517$&\literal{pc}&\includeMPgraphics{figure-dimension-pc}\\
\small ци́церо&$12{,}79213$&\literal{cc}&\includeMPgraphics{figure-dimension-cc}\\
\small сантиметр&$28{,}34645$&\literal{cm}&\includeMPgraphics{figure-dimension-cm}\\
\small дюйм&$72$&\literal{in}&\includeMPgraphics{figure-dimension-in}
\end{tabular}
\end{center}
\end{frame}

%%%%%%%%%%

\begin{frame}{Числа: арифметика и~тригонометрия}
\centering
\begin{tabular}{r@{\enspace→\enspace}l}
\literal{+\,}$x$
	&$+x$\\
\literal{-\,}$x$
	&$-x$\\
$x$\literal{\,+\,}$y$
	&$x+y$\\
$x$\literal{\,-\,}$y$
	&$x-y$\\
$x$\literal{\,*\,}$y$
	&$xy$\\
$x$\literal{\,/\,}$y$
	&$x\fracslash y$\\
$x$\literal{\,**\,}$y$
	&$x^y$\\
$x$\literal{\,++\,}$y$
	&$\sqrt{x^2+y^2}$\\
$x$\literal{\,+-+\,}$y$
	&$\sqrt{x^2-y^2}$\\
\literal{sqrt~}$x$
	&$\sqrt x$\\
\literal{abs~}$x$
	&$|x|$
\end{tabular}
%\enspace\vrule\enspace
~\vrule~%
\begin{tabular}{r@{\enspace→\enspace}l}
\literal{floor~}$x$
	&целая часть~$x$\\
\literal{round~}$x$
	&округление~$x$\\
$x$\literal{~mod~}$y$
	&остаток от деления~$x$ на~$y$\\
$x$\literal{~div~}$y$
	&целая часть отношения~$x$ и~$y$\\
\literal{cosd~}$\theta$
	&$\cos(\theta\cdot\slashfrac\pi{180})$\\
\literal{sind~}$\theta$
	&$\sin(\theta\cdot\slashfrac\pi{180})$\\
$t$\literal[$a$\literal{,\,}$b$\literal]
	&медиатор \textit{(см.~дальше)}
\end{tabular}
\end{frame}

%%%%%%%%%%

\begin{frame}{Пары: декомпозиция и~арифметика}
\centering
\begin{tabular}{r@{\enspace→\enspace}l}
\literal{xpart~(}$x$\literal{,\,}$y$\literal{)}
	&$x$\\
\literal{ypart~(}$x$\literal{,\,}$y$\literal{)}
	&$y$\\
\literal{(}$x$\literal{,\,}$y$\literal{)\,+\,(}$u$\literal{,\,}$v$\literal{)}
	&\literal{(}$x+u$\literal{,\,}$y+v$\literal{)}\\
$t$\literal{\,*\,(}$x$\literal{,\,}$y$\literal{)},
	\literal{(}$x$\literal{,\,}$y$\literal{)\,*\,}$t$
	&\literal{(}$tx$\literal{,\,}$ty$\literal{)}\\
\literal{(}$x$\literal{,\,}$y$\literal{)\,/\,}$t$
	&\literal{(}$\slashfrac xt$\literal{,\,}$\slashfrac yt$\literal{)}\\
\literal{(}$x$\literal{,\,}$y$\literal{)~dotprod~(}$u$\literal{,\,}$v$\literal{)}
	&$xu+yv$\\
\literal{abs~(}$x$\literal{,\,}$y$\literal{)}
	&$\sqrt{x^2+y^2}$\\
\literal{unitvector~(}$x$\literal{,\,}$y$\literal{)}
	&\literal{(}$x$\literal{,\,}$y$\literal{)\,/\,abs~(}$x$\literal{,\,}$y$\literal{)}\\
\literal{angle~(}$x$\literal{,\,}$y$\literal{)}
	&$\slashfrac{180}\pi\cdot\arctg(\slashfrac yx)$\\
\literal{dir~}$\theta$
	&\literal{(}$\cos(\theta\cdot\slashfrac\pi{180})$%
	\literal{,\,}$\sin(\theta\cdot\slashfrac\pi{180})$\literal{)}
\end{tabular}
\end{frame}

%%%%%%%%%%

\begin{frame}{Медиатор}
Для числа~$t$ и~величин $a$~и~$b$ одинакового типа, либо \literal{numeric},
либо \literal{pair}, либо \literal{color}, в~\hologo{METAPOST} определяется
операция \emph{медиатор (аффинная комбинация)}:

\begin{center}
\LARGE $t$\literal[$a$\literal{,~}$b$\literal]\quad→\quad$(1-t)\,a+t\,b$
\end{center}
\end{frame}

%%%%%%%%%%

\begin{frame}{Медиатор}
\centering
\LARGE
\literal{.2[3, 8]}\quad→\quad\literal4\\[3ex]
\literal{.6[z1, z2]}\\
↓\\
{\normalsize точка, делящая отрезок от \literal{z1} до \literal{z2}
в~отношении~$3:2$}
\end{frame}

%%%%%%%%%%

\begin{frame}{Кинематический смысл медиатора}
\begin{columns}
\column{.3\textwidth}
\includeMPgraphics{figure-mediator-kin}
\column{.7\textwidth}
Между прочим, выражение
	\[
	z(t)=(1-t)z_0+tz_1
	\]
задаёт закон равномерного прямолинейного движения, при котором точка
с~радиусом"=вектором~$z(t)$ в~момент времени~$t=0$ проходит через точку~$z_0$,
а~при~$t=1$~— через~$z_1$.

\bigskip

Отрицательные значения~$t$ соответствуют точкам, находящимся на продолжении
отрезка~$z_0z_1$ за точку~$z_0$, а~значения, бо́льшие единицы, отвечают точкам
на продолжении этого отрезка за точку~$z_1$.
\end{columns}
\end{frame}
