\sectionwithlogo{Макрокоманды}
	{\scalebox{.9}{\input{figure-tree-10}}}

\begin{frame}{Метод декомпозиции в~программировании}
В~соответствии с~методом декомпозиции программист стремится разбивать
исходную, стоящую перед ним задачу на более простые подзадачи. Так~же он
поступает с~подзадачами до тех пор, пока наиболее мелкие подзадачи не станут
предельно простыми.

Тем самым программист добивается следующего:
\begin{itemize}
\item
появляется возможность коллективной работы над задачей, распределения работы
между сотрудниками, каждый из которых занят решением своей подзадачи;

\item
будущая программа перестаёт быть аморфной и~приобретает структуру, в~результате
работа над ней упрощается;

\item
если для подзадач имеются готовые (возможно, чужие) решения, их можно
использовать, сократив интеллектуальные издержки.
\end{itemize}
\end{frame}

\begin{frame}{Метод декомпозиции и~процедуры}
Во~многих алгоритмических языках для поддержки метода декомпозиции
предусмотрены процедуры (функции, подпрограммы).

Программа представляется в~виде последовательности вызовов процедур: главная
процедура вызывает процедуры, решающие подзадачи второго уровня; те, в~свою
очередь, вызывают процедуры третьего уровня, и~т.~д.
\end{frame}

\begin{frame}{Принцип повторного использования кода}
Принцип повторного использования кода в~программировании требует, чтобы
программист вычленял в~исходной задаче группы однотипных подзадач.

Готовые подпрограммы для решения подзадач могут вызываться многократно.

За счёт этого код программы становится короче, яснее, его легче отлаживать
(замеченная неточность в~подпрограмме исправляется один раз, а не во~многих
местах программного кода).

Общезначимые процедуры могут быть собраны в~библиотеки, которые могут быть
подключены не только к~данной программе, но и~к~другим.
\end{frame}

\begin{frame}{Макрокоманды}
В~\hologo{METAPOST} аналогами процедур являются \emph{макрокоманды}.

Несмотря на некоторое отличие от процедур по способу реализации, макрокоманды
играют ту~же роль, и~используются сходным образом.

Макрокоманды, как и~процедуры, могут принимать параметры. Тогда они могут
решать не одну единственную задачу, а~класс подзадач.
\end{frame}

\begin{frame}{Макрокоманды без параметров}
Синтаксис:
\begin{center}
\LARGE
\literal{def~}\replaceable{имя}\literal{=}\replaceable{тело макрокоманды}\literal{~enddef}
\end{center}

Например,
\begin{programlisting}
\Large
def \alert{\only<1>{up}\only<2>{down}to}=\par
~~~~\alert{step \only<2>{-}1 until}\par
enddef;
\end{programlisting}
\end{frame}

\begin{frame}{Макрокоманды с~параметрами}
Синтаксис:
\begin{center}
\LARGE
\literal{def~}\replaceable{имя}\literal{(expr~}\replaceable{список параметров}\literal{)=}\\
\replaceable{тело макрокоманды} \literal{~enddef}
\end{center}

Например,
\begin{programlisting}
\Large
def \alert{rotatedabout}(expr \alert{z, d})=\par
~~~~\alert{shifted -z rotated d shifted z}\par
enddef;
\end{programlisting}
\end{frame}

\begin{frame}{Библиотека \nolinkurl{plain.mp}}
Приведённые определения макрокоманд взяты из файла \nolinkurl{plain.mp}, который
\hologo{METAPOST} прочитывает перед началом работы.

Таким образом, многие полезные макрокоманды становятся доступными для нас.

Кроме определений макрокоманд, файл содержит и другие команды, приводящие
\hologo{METAPOST} в~рабочее состояние.

Например, там объявляются и~получают значения полезные переменные, вроде
\literal{cm}, \literal{origin}, \literal{right}, \literal{up}, \literal{left},
\literal{down}, \literal{red}, \literal{green}, \literal{blue},
\literal{mitered}, \literal{rounded},~…
\end{frame}

\begin{frame}{Библиотеки}
Возможно создавать собственные библиотеки.

Для этого нужные команды, тематически сгруппированные, следует поместить в~файл,
скажем, \nolinkurl{planimetry.mp}, поместить этот файл в~такое место, где
его найдёт \hologo{METAPOST}, например, в~текущую директорию.

Затем в~наших программах (или в~других библиотеках) можно подключить эту
библиотеку командой
\begin{programlisting}
input planimetry;
\end{programlisting}
\end{frame}

\begin{frame}{Группировка и~локализация переменных}
Группировка~— это способ объединения нескольких команд в~одну, составную. При
этом изменения переменных, которые происходят внутри группы, можно
локализовать.

Это значит, что изменения переменных, произошедшие в~группе, будут забыты после
её завершения (закрытия), если, конечно, программист позаботится о~локальности
этих переменных.

Для группировки необходимо поместить последовательность команд (после каждой~—
точка с~запятой) между словами \literal{begingroup} и~\literal{endgroup}:
\begin{center}
\LARGE
\literal{begingroup~}\replaceable{команды}\literal{~endgroup}
\end{center}
\end{frame}

\begin{frame}{Группировка и~локализация переменных}
Для того, чтобы сделать переменные локальными в~группе, нужно временно
сохранить прежние значения командой \literal{save}:
\begin{programlisting}
begingroup\par
~~~~\alert{save a, b, c};\par
~~~~numeric a, b;\par
~~~~pair c;\par
~~~~…\par
endgroup
\end{programlisting}
\end{frame}

\begin{frame}{Группы как функции}
У~группировки имеется и~другое предназначение. Если группа завершается
выражением (то есть в~конце группы отсутствует точка с~запятой)
\hologo{METAPOST} подставляет вместо группы значение выражения:
\begin{center}
\LARGE
\literal{begingroup~}\replaceable{команды}~\replaceable{выражение}\literal{~endgroup}
\end{center}

Обычно \replaceable{команды} посвящены закулисным делам, которые нужны для
вычисления \replaceable{выражения}.
\end{frame}

\begin{frame}{Пример: вычисление факториала}
Допустим, в~программе потребовалось где"=то значение $6!$. Конечно, можно прямо
написать~$720$, но мы на этом примере проиллюстрируем группировку, чтобы
в~более трудных случаях действовать по аналогии.
\begin{programlisting}
numeric six\_factorial;\par
six\_factorial=begingroup\par
~~~~save f;\par
~~~~numeric f;\par
~~~~f:=1;\par
~~~~for i:=2 upto 6: f:=f*i; endfor;\par
~~~~\alert{f}~~~~\textcolor{cyan}{\% нет точки с~запятой!}\par
endgroup;
\end{programlisting}
\end{frame}

\begin{frame}{Пример: вычисление факториала}
Благодаря нашим заботам приключения переменной \literal{f} внутри группы никак
не повлияют на внешний мир.

В~частности, если снаружи группы уже была объявлена такая переменная, то она
сохранит своё прежнее значение и~тип (который мог отличаться от
\literal{numeric}).

Если~же переменная \literal{f} вне группы не определялась, такое~же положение
дел сохранится и~после приведённого кода.
\end{frame}

\begin{frame}{Макрокоманда для факториала}
Чтобы сделать код для вычисления факториала более универсальным (пригодным для
вычисления не только~$6!$, но и~$n!$ для произвольного~$n$), оформим его как
макрокоманду с~параметром:
\begin{programlisting}
def factorial(expr \alert{n})=\par
~~~~begingroup\par
~~~~~~~~save f;\par
~~~~~~~~numeric f;\par
~~~~~~~~f:=1;\par
~~~~~~~~for i:=2 upto \alert{n}: f:=f*i; endfor;\par
~~~~~~~~f\par
~~~~endgroup~~~~\textcolor{cyan}{\% нет точки с запятой!}\par
enddef;
\end{programlisting}
\end{frame}

\begin{frame}{Макрокоманда для факториала}
Обратите внимание на отсутствие точки с~запятой после \literal{endgroup}. Если
бы она была там, команда
\begin{programlisting}
s:=factorial(5)+factorial(6);
\end{programlisting}
воспринималось~бы как
\begin{programlisting}
s:=120\alert{;}+720\alert{;};
\end{programlisting}
Это~бы привело, во"=первых, к~неправильному значению \literal{s}, и,
во"=вторых, к~ошибке:
\begin{screen}
>> 720\par
\only<1>{! Isolated expression.}%
\only<2>{! Изолированное выражение.}%
\transdissolve<2>
\end{screen}
\end{frame}

\begin{frame}{Изолированное выражение}
Ошибка «изолированное выражение» является довольно распространённой и обычно
свидетельствует о~неполадках с~точками с~запятой.
\begin{programlisting}
s:=120;\alert{+720\alert;};
\end{programlisting}
В~нашем примере проблемной будет выделенная команда.

Выражение \literal{+720} будет вычислено (значение~$720$), и~затем никак не
использовано. Это является недопустимым в~\hologo{METAPOST}, хотя в~некоторых
алгоритмических языках это не криминал.
\end{frame}

\begin{frame}{Макрокоманды \literal{vardef}}
Поскольку часто определение макрокоманды сочетается с~группировкой,
в~\hologo{METAPOST} имеется вариант макроопределений~— \literal{vardef}. В~теле
таких определений группировка уже предполагается неявно:
\begin{programlisting}
\alert{\only<2-3>{var}def} factorial(expr n)=\par
\only<1-2>{\uncover<1>{~~~~\alert{begingroup}\par}}%
\only<1-2>{~~~~}~~~~save f;\par
\only<1-2>{~~~~}~~~~numeric f;\par
\only<1-2>{~~~~}~~~~f:=1;\par
\only<1-2>{~~~~}~~~~for i:=2 upto n: f:=f*i; endfor;\par
\only<1-2>{~~~~}~~~~f\par
\only<1-2>{\uncover<1>{~~~~\alert{endgroup}\par}}%
enddef;
\end{programlisting}
\end{frame}

\begin{frame}{Пример: пересечение прямых}
Оформим код для нахождения пересечения прямых как макрокоманду. На вход
макрокоманда будет получать четыре точки (пары), на выходе будет возвращать
точку пересечения прямых, проходящих через первые две (\literal{p}
и~\literal{q}) и~последние две (\literal{r} и~\literal{s}) точки.
\begin{programlisting}
vardef intersection(expr p, q, r, s)=\par
~~~~save z;\par
~~~~pair z;\par
~~~~z=whatever[p, q]=whatever[r, s];\par
~~~~z\par
enddef;
\end{programlisting}
\end{frame}

\begin{frame}{Определение макрокоманды \literal{whatever}}
В~библиотеке \nolinkurl{plain.mp} команда \literal{whatever} определяется так:
\begin{programlisting}
vardef whatever=save ?; ? enddef;
\end{programlisting}
В~теле этого макроопределения локализуется переменная \literal{?}
(в~\hologo{METAPOST} возможны и~такие имена), и, не получая никакого значения,
возвращается. Имя этой переменной тут~же забывается, а~неопределённое значение
можно использовать.

Таким образом, команда \literal{whatever}~— это своего рода генератор анонимных
числовых переменных.
\end{frame}

\begin{frame}{Пример: описанная окружность}
Определяем центр описанной окружности как точку пересечения серединных
перпендикуляров (медиатрис) сторон треугольника:
\begin{programlisting}
vardef circumcenter(expr p, q, r)=\par
~~~~intersection(\par
~~~~~~~~.5[p, q],\par
~~~~~~~~.5[p, q]+((q-p) rotated 90),\par
~~~~~~~~.5[q, r],\par
~~~~~~~~.5[q, r]+((r-q) rotated 90)\par
~~~~)\par
enddef;
\end{programlisting}
\end{frame}

\begin{frame}{Пример: описанная окружность}
Радиус находим как расстояние от центра до одной из вершин (неважно какой):
\begin{programlisting}
vardef circumradius(expr p, q, r)=\par
~~~~abs(circumcenter(p, q, r)-p)\par
enddef;
\end{programlisting}

Сама окружность (путь):
\begin{programlisting}
vardef circumcircle(expr p, q, r)=\par
~~~~fullcircle\par
~~~~~~~~scaled (2circumradius(p, q, r))\par
~~~~~~~~shifted circumcenter(p, q, r)\par
enddef;
\end{programlisting}
\end{frame}

\begin{frame}{Пример: вписанная окружность}
Центр вписанной в~треугольник окружности найдём как точку пересечения двух
биссектрис. Воспользуемся тем, что сумма двух единичных векторов служит
биссектрисой угла между этими векторами, так~что биссектриса угла~\literal{a}
треугольника проходит через точку \literal{a+unitvector(b-a)+unitvector(c-a)}
(\literal{b} и~\literal{c}~— остальные вершины).
\begin{programlisting}
vardef incenter(expr a, b, c)=\par
~~~~intersection(\par
~~~~~~~~a, a+unitvector(b-a)+unitvector(c-a),\par
~~~~~~~~b, b+unitvector(c-b)+unitvector(a-b)\par
~~~~)\par
enddef;
\end{programlisting}
\end{frame}

\begin{frame}{Пример: вписанная окружность}
\begin{programlisting}
vardef inradius(expr a, b, c)=\par
~~~~save i;\par
~~~~pair i;\par
~~~~i=incenter(a, b, c);\par
~~~~abs(i-intersection(a, b, i, i+((b-a) zscaled up)))\par
enddef;
\end{programlisting}

\begin{programlisting}
vardef incircle(expr a, b, c)=\par
~~~~fullcircle scaled 2inradius(a, b, c)\par
~~~~~~~~shifted incenter(a, b, c)\par
enddef;
\end{programlisting}
\end{frame}


\begin{frame}{Использование макрокоманды \literal{circumcircle}}
\begin{columns}
\column{.4\textwidth}
\input{figure-circumcircle}
\column{.6\textwidth}
\begin{programlisting}
z1=origin;\par
z2=(2cm, 0);\par
z3=(.5cm, 3cm);\par
draw z1--z2--z3--cycle;\par
draw circumcircle(z1, z2, z3)\par
~~~~withcolor red;
\end{programlisting}
\end{columns}
\end{frame}

\begin{frame}{Рекурсия}
В~\hologo{METAPOST} разрешена \emph{рекурсия}, то~есть ситуация, когда
макрокоманда вызывает сама себя, прямо или косвенно.

Проиллюстрируем этот тезис на классическом примере~— вычислении факториала.
\end{frame}

\begin{frame}{Рекурсивное вычисление факториала}
Воспользуемся известным соотношением $n!=n(n-1)!$. Вместе с~равенством $0!=1$
эта формула даёт способ рекурсивного вычисления факториала:

\begin{programlisting}
def factorial(expr n)=\par
~~~~if n=0: 1 else: n*factorial(n-1) fi\par
enddef;
\end{programlisting}
\end{frame}

\begin{frame}{Дерево}
\centerline{\input{figure-tree-10}}
\end{frame}

\begin{frame}{Дерево как рекурсивная структура}
Наблюдения за деревьями, этими прекрасными творениями природы, показывают, что
они имеют рекурсивное устройство.

Дерево состоит из~ствола и~растущих из него ветвей, каждая из~которых является
уменьшенной копией дерева.

Такое определение дерева предусматривает бесконечное количество ветвей. Оно
нуждается в~уточнении.

Дерево $n$-го порядка (при $n>0$) состоит из~ствола и~растущих из~него ветвей,
каждая из~которых представляет собой дерево $(n-1)$-го порядка. Дерево $0$-го
порядка~— это голый ствол.
\end{frame}

\begin{frame}{Рост дерева:
\only<1>{0}%
\only<2>{1}%
\only<3>{2}%
\only<4>{3}%
\only<5>{4}%
\only<6>{5}%
\only<7>{6}%
\only<8>{7}%
\only<9>{8}%
\only<10>{9}%
\only<11>{10}%
-й порядок}
\centerline{%
\only<1>{\input{figure-tree-0}}%
\only<2>{\input{figure-tree-1}}%
\only<3>{\input{figure-tree-2}}%
\only<4>{\input{figure-tree-3}}%
\only<5>{\input{figure-tree-4}}%
\only<6>{\input{figure-tree-5}}%
\only<7>{\input{figure-tree-6}}%
\only<8>{\input{figure-tree-7}}%
\only<9>{\input{figure-tree-8}}%
\only<10>{\leavevmode
\begin{mplibcode}
input tree;

beginfig(0)

randomseed:=.1;
unfill unitsquare xscaled 6.5cm yscaled 4.75cm shifted (-3.25cm, -.2cm);
tree(origin, (0, cm), 9, 2mm, .25green);

endfig;
\end{mplibcode}
}%
\only<11>{\input{figure-tree-10}}%
}
\end{frame}

\begin{frame}{Построение дерева}
\begin{columns}
\column{.4\textwidth}
\leavevmode
\begin{mplibcode}
input tree;

beginfig(0)

randomseed:=.1;
thickness=4mm;
z1=origin;
z2=(0, 2cm);
z3=shortening[z2, z1 rotatedaround(z2, offset+uniformdeviate(branchrotation))];
z4=shortening[z2, z1 rotatedaround(z2, -offset-uniformdeviate(branchrotation))];

% TODO markangle.arcfill(z2+dir 45, z2, z2+up) extra=>80, fg=>.25red;

draw z1--z2 withpen pencircle scaled thickness withcolor .5green;
draw z2--z3 withpen pencircle scaled (thinning*thickness)
	withcolor lightening[.5green, white];
draw z2--z4 withpen pencircle scaled (thinning*thickness)
	withcolor lightening[.5green, white];
draw z2--(z2+3cm*dir(45))
	withcolor .5red
	withpen tinypen;
draw (z2+3cm*up)--(z2+1.5cm*unitvector(z1-z2))
	withcolor .5red
	withpen tinypen;

draw z1 withpen pencircle scaled 2bp withcolor .5red;
draw z2 withpen pencircle scaled 2bp withcolor .5red;
draw z3 withpen pencircle scaled 2bp withcolor .5red;
draw z4 withpen pencircle scaled 2bp withcolor .5red;

markangle.arcarrowboth.ccw(z2, z2+dir 45, z2+up, 1) fg=>red, scale=>9; % TODO extra=>80, fg=>.5red;
%markangle.arcdblarrow(z2+down, z2, z2+dir 45, 1) extra=>40, fg=>.5red;
%markangle.arc(z3, z2, z2+up, 1) extra=>10;

%begingroup
interim labeloffset:=8bp;

label.bot("$a$", z1);
label.lft("$b$", z2);
label.top("$c$", z3);
label.top("$d$", z4);
label.top("$\theta$", z2+2.2cm*dir 62);
%label.rt("$\phi$", z2+cm*dir -22.5);
%endgroup

endfig
\end{mplibcode}

\column{.6\textwidth}
\only<1>{%
Правая ветка дерева отклоняется от вертикали в~пределах закрашенного сектора.
Угол отклонения случайный и~равномерно распределённый в~пределах от~$0$
до~$\theta$ (в~нашем примере~$45^\circ$.

\bigskip

Угол~$\theta$ будет в~программе храниться в~переменной \literal{theta}.}%
\only<2>{%
Длина веток получается умножением параметра \literal{shortening} на длину
ствола дерева.

\bigskip

Толщина веток получается умножением параметра \literal{thinning} на толщину
ствола дерева \literal{thickness}.

\bigskip

Ветки светлее ствола, и~параметр \literal{lightening} отвечает за осветление.}

\end{columns}
\end{frame}

\begin{frame}{Дерево: код программы}
Определяем параметры…
\begin{programlisting}
theta:=45;\par
thinning:=.7;\par
shortening:=.8;\par
lightening:=.1;
\end{programlisting}
\end{frame}

\begin{frame}{Дерево: код программы}
Определяем макрокоманду \literal{tree} с~параметрами \literal{a} и~\literal{b}
(основание и~верхушка ствола), \literal{n} (порядок дерева),
\literal{thickness} (толщина ствола), \literal{clr} (цвет ствола).
\begin{programlisting}
vardef tree(expr a, b, n, thickness, clr)=\par
~~~~…\par
enddef;
\end{programlisting}
(код макрокоманды на следующем слайде).

Вызываем макрокоманду \literal{tree}:
\begin{programlisting}
tree(origin, (0, cm), 10, 2mm, .25green);
\end{programlisting}

Рисунок готов.
\end{frame}

\begin{frame}{Тело макрокоманды \literal{tree}}
Объявляем параметры~— координаты верхушек веток…
\begin{programlisting}
save c, d;\par
pair c, d;
\end{programlisting}

Определяем координаты верхушки правой ветки…
\begin{programlisting}
c:=shortening[b,\par
~~~~a rotatedaround(b, 180-theta+uniformdeviate(theta))];
\end{programlisting}
и~левой…
\begin{programlisting}
d:=shortening[b,\par
~~~~a rotatedaround(b, theta-180+uniformdeviate(theta))];
\end{programlisting}
\end{frame}

\begin{frame}{Тело макрокоманды \literal{tree}}
Рисуем ствол…
\begin{programlisting}
draw a--b\par
~~~~withpen pencircle scaled thickness\par
~~~~withcolor clr;
\end{programlisting}

Для деревьев порядка ${}>0$ рисуем ветки…
\begin{programlisting}
if n>0:\par
~~~~tree(b, c, n-1, thinning*thickness,\par
~~~~~~~~lightening[clr, white]);\par
~~~~tree(b, d, n-1, thinning*thickness,\par
~~~~~~~~lightening[clr, white]);\par
fi
\end{programlisting}
\end{frame}

\begin{frame}{Макрокоманды \literal{vardef … @\#}}
В~библиотеке \nolinkurl{plain.mp} даётся следующее определение:
\begin{programlisting}
vardef z@\#=(x@\#, y@\#) enddef;
\end{programlisting}
Такое определение создаёт видимость того, что переменные, чьи имена начинаются
с~токена \literal{z}, объявлены как переменные типа \literal{pair}.

Таким образом, можно пользоваться «переменными» с~именами \literal{z1},
\literal{z13a}, \literal{z.p.q[25]} как будто они объявлены как пары.

На самом деле при употреблении таких имён происходит подстановка тела
макрокоманды \literal{z@\#}: вместо \literal{z.p.q[25]} подставляется
\literal{(x.p.q[25], y.p.q[25])}. Переменные \literal{x.p.q[25]}
и~\literal{y.p.q[25]}, если они не объявлялись, воспринимаются в~программе как
числовые (\literal{numeric}), что и~требуется.
\end{frame}

\begin{frame}{Макрокоманды \literal{vardef … @\#}}
Кроме всего прочего, в~определении макрокоманды \literal{beginfig} имеется
фрагмент
\begin{programlisting}
begingroup\par
save x, y;\par
\end{programlisting}
а~в~определении команды \literal{endfig}~—
\begin{programlisting}
endgroup
\end{programlisting}
\end{frame}

\begin{frame}{Макрокоманды \literal{vardef … @\#}}
Так~что в~начале каждого рисунка переменные, чьи имена начинаются с~токенов
\literal{x} или \literal{y}, становятся независимыми от таких~же имён внутри
кода других рисунков.

Код многих рисунков, как мы видели в~примерах, начинается сразу с~определений
пар \literal{z.}\replaceable{суффикс} (а~их объявление пропускается):
\begin{programlisting}
z1=origin;\par
z2=(3cm, 0);\par
z3=(0, 4cm);\par
\leavevmode\par
draw z1--z2--z3--cycle;\par
\end{programlisting}
\end{frame}

\begin{frame}{Применение определений \literal{vardef … @\#}}
Эта особенность определений \literal{vardef} позволяет передавать
в~макрокоманды не только выражения, но и~имена.

Например, можно определить макрокоманду, которая построит и~возвратит график
функции (путь) на заданном отрезке. Она в~качестве параметров примет числа~—
концы заданного отрезка (\literal{a} и~\literal{b}), шаг \literal{delta},
а~также имя функции (\literal{@\#}), чей график следует построить.

Определение будет таким:
\begin{programlisting}
vardef graph@\#(expr a, b, delta)=\par
~~~~for i:=a step delta until b:\par
~~~~~~~~(i, @\#(i))--\par
~~~~endfor (b, @\#(b))\par
enddef;
\end{programlisting}
\end{frame}

\begin{frame}{Применение макрокоманды \literal{graph}}
В~качестве примера применения макрокоманды \literal{graph} построим графики
функций $f(x)=x^x$ и~$g(x)=\sqrt{1-x^2}$ на отрезке $[0;1]$ с~шагом $0{,}01$
(полученная ломаная будет иметь сто звеньев). Код для рисования координатных
осей не приводится.

\begin{columns}
\column{.4\textwidth}
\centerline{\leavevmode
\begin{mplibcode}
beginfig(0)

save f, g;

vardef f(expr x)=x**x enddef;
vardef g(expr x)=1+-+x enddef;
vardef graph@#(expr a, b, delta)=
	for i:=a step delta until b:
		(i, @#(i))--
	endfor
	(b, @#(b))
enddef;

drawaxes(-.25cm, -.25cm, 3.25cm, 3.25cm);
draw graph.f(0, 1, .01) scaled 3cm withcolor red;
draw graph.g(0, 1, .01) scaled 3cm withcolor .5green;

label.bot(btex $x$ etex, (3.25cm, 0));
label.lft(btex $y$ etex, (0, 3.25cm));

endfig
\end{mplibcode}
}
\column{.6\textwidth}
\begin{programlisting}
vardef f(expr x)=x**x enddef;\par
vardef g(expr x)=1+-+x enddef;\par
\leavevmode\par
draw graph.f(0, 1, .01)\par
~~~~scaled 3cm\par
~~~~withcolor red;\par
draw graph.g(0, 1, .01)\par
~~~~scaled 3cm\par
~~~~withcolor .5green;
\end{programlisting}
\end{columns}
\end{frame}

\begin{frame}{Нелинейные уравнения}
Некоторые построения требуют решения нелинейных уравнений.

В~\hologo{METAPOST} отсутствуют встроенные средства для этого.
\end{frame}

\begin{frame}{Процедура \literal{solve}}
Процедура \literal{solve} из библиотеки \nolinkurl{plain.mp} позволяет численно
решать уравнения с~одной неизвестной методом бисекций (бинарным поиском).

\begin{center}
\LARGE
\literal{solve~}$f$\literal{(}$t_+$\literal{,}$t_-$\literal{)}
\end{center}

Здесь $f$~— имя процедуры с~одним числовым параметром, возвращающей значение
типа \literal{boolean}. Её мы должны запрограммировать сами с~помощью
\literal{vardef}.

Процедура \literal{solve} возвращает число $t\in[t_+,t_-]$, близко к~которому
значение $f(t)$ меняются с~истинного на ложное. Предполагается, что $f(t_+)$
истинно, а~$f(x_-)$~— ложно.
\end{frame}

\begin{frame}{Пример с~нелинейным уравнением}
\centerline{\leavevmode
\begin{mplibcode}
input elementi;

beginfig(0)

u:=cm;
z1=origin;
z2=(5u, 0);
z3=(1.5u, 3u);

vardef f(expr t)=
	inradius(z1, t[z1, z2], z3)<inradius(t[z1, z2], z2, z3)
enddef;

z4=(solve f(0, 1))[z1, z2];

draw z1--z2--z3--cycle;
draw z4--z3;

draw incircle(z1, z4, z3);
draw incircle(z4, z2, z3);

%label.bot("$a$", z1);
%label.bot("$b$", z2);
%label.top("$c$", z3);
%label.bot("$d$", z4);

endfig
\end{mplibcode}
}

Требуется провести чевиану в~треугольнике так, чтобы она разделила его на два
треугольника с~равными радиусами вписанных кругов.

Задача сводится к~решению системы рациональных уравнений. Общее решение можно
найти вручную (очень сложно) или с~помощью систем компьютерной алгебры. Для
наших целей подойдёт численное решение.
\end{frame}

\begin{frame}{Пример с~нелинейным уравнением {\mdseries\itshape(продолжение)}}
Определим вершины треугольника:
\begin{programlisting}
z1=origin;\par
z2=(5cm, 0);\par
z3=(1.5cm, 3cm);
\end{programlisting}

Искомое основание чевианы будем искать как точку $t[z_1,z_2]$, где~$t$ найдём
с~помощью \literal{solve}.
\end{frame}

\begin{frame}{Пример с~нелинейным уравнением {\mdseries\itshape(продолжение)}}
Определим~процедуру~\literal{f}, используя ранее определённую процедуру
\literal{inradius}:
\begin{programlisting}
vardef f(expr t)=\par
~~~~inradius(z1, t[a, b], z3)<inradius(t[z1, z2], z2, z3)\par
enddef;
\end{programlisting}

При $t_+=0$ она возвращает истину, при $t_-=1$~— ложь. Где-то между этим
значениями истинность $f(t)$ меняется. Теперь находим основание чевианы:
\begin{programlisting}
z4=(solve f(0, 1))[z1, z2];
\end{programlisting}
\end{frame}

\begin{frame}{Пример с~нелинейным уравнением {\mdseries\itshape(продолжение)}}
\begin{columns}
\column{.45\textwidth}
Рисуем:
\begin{programlisting}
draw z1--z2--z3--cycle;\par
draw z4--z3;\par
draw incircle(z1, z4, z3);\par
draw incircle(z4, z2, z3);\par
\end{programlisting}
\column{.45\textwidth}
\centerline{\leavevmode
\begin{mplibcode}
input elementi;

beginfig(0)

u:=cm;
z1=origin;
z2=(5u, 0);
z3=(1.5u, 3u);

vardef f(expr t)=
	inradius(z1, t[z1, z2], z3)<inradius(t[z1, z2], z2, z3)
enddef;

z4=(solve f(0, 1))[z1, z2];

draw z1--z2--z3--cycle;
draw z4--z3;

draw incircle(z1, z4, z3);
draw incircle(z4, z2, z3);

%label.bot("$a$", z1);
%label.bot("$b$", z2);
%label.top("$c$", z3);
%label.bot("$d$", z4);

endfig
\end{mplibcode}
}
\end{columns}
\end{frame}
