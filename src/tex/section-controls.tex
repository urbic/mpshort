\sectionwithlogo{Управляющие конструкции}
	%{\scalebox{.8}{\includeMPgraphics{figure-for-3}}}
	{\includeMPgraphics{figure-for-3}}

\begin{frame}{Управляющие конструкции в~\hologo{METAPOST}}
В~\hologo{METAPOST}, как и~в~других императивных языках, имеются конструкции,
влияющие на порядок выполнения команд: \emph{условная конструкция}
и~\emph{цикл}.
\end{frame}

\begin{frame}{Цикл \literal{for}}
\begin{columns}
\column{.4\textwidth}
\centering
\only<1>{\includeMPgraphics{figure-for-1}}%
\only<2>{\includeMPgraphics{figure-for-2}}%
\only<3>{\includeMPgraphics{figure-for-3}}%
\column{.6\textwidth}
\begin{programlisting}
for i:=0 step 1 until 10:\par
\only<1,3>{~~~~draw \only<1,3>{origin--3cm*dir(9i)};\par}%
\only<2,3>{~~~~draw quartercircle\par~~~~~~~~scaled (6cm*i/10);\par}%
endfor
\end{programlisting}
\end{columns}
\end{frame}

\begin{frame}{Сокращения \literal{upto} и~\literal{downto}}
В~циклах \literal{for} с~шагом~\literal1 можно воспользоваться удобным сокращением
(макрокомандой) \literal{upto}:
\begin{programlisting}
for i:=0 \alert{\only<1>{step 1 until}\only<2>{upto}} 10:\par
~~~~…\par
endfor
\end{programlisting}

Для фразы \literal{step -1 until} имеется сокращение \literal{downto}.
\end{frame}

\begin{frame}{Цикл \literal{for} с~явным перечислением}
Рассмотренные ранее циклы \literal{for} по очереди присваивали переменной цикла
значения из некоторой арифметической прогрессии.

Есть возможность явным образом перечислить значения, которые будут перебираться
в~цикле:
\begin{programlisting}%
for i:=2, 3, 5, 7, 11, 13, 17, 19, 23:\par
~~~~…\par
endfor;
\end{programlisting}
\end{frame}

\begin{frame}{Цикл \literal{for} с~явным перечислением}
Перечисляемые значения не обязаны быть одного типа:
\begin{programlisting}
for i:=2, 2.71828459045, "METAPOST", up:\par
~~~~show(i);\par
endfor;
\end{programlisting}

Этот код приведёт к~выводу всех перечисленных разнотипных значений на терминал
и~в~файл протокола:
\begin{screen}%
>> 2\par
>> 2.71828\par
>> "METAPOST"\par
>> (0,1)
\end{screen}
\end{frame}

\begin{frame}{Особенности цикла \literal{for}}
Тело цикла \literal{for} не обязательно должно быть законченной командой. При
обнаружении цикла \literal{for} компилятор \hologo{METAPOST} производит
многократное повторение тела цикла с~подстановкой вместо имени переменной цикла
(\literal{i}) её текущего, изменяющегося в~цикле, значения. Таким образом,
команда
\begin{programlisting}
s:=for i:=1 upto 100: \alert{i+} endfor 0;
\end{programlisting}
воспринимается как
\begin{programlisting}
s:=\alert{1+2+3+4+5+…+95+96+97+98+99+100+}0;
\end{programlisting}
\end{frame}

\begin{frame}{Особенности цикла \literal{for}}
Впрочем, возможно и~«классическое» использование цикла для подсчёта суммы чисел
от~$1$ до~$100$:
\begin{programlisting}%
s:=0;\par
for i:=1 upto 100: \alert{s:=s+i;} endfor;
\end{programlisting}
Этот код интерпретируется как
\begin{programlisting}%
s:=0;\par
\alert{s:=s+1;\par
s:=s+2;\par
…\par
s:=s+99;\par
s:=s+100;}
\end{programlisting}
\end{frame}

\begin{frame}{Построение путей в~цикле}
Упомянутая особенность цикла \literal{for} позволяет строить пути в~цикле.

Рассмотрим эту возможность на примере графика функции $y=\cos x^2$. График
будет представлен как ломаная.
\begin{center}
\includeMPgraphics{figure-graph-cossq}
\end{center}
\end{frame}

\begin{frame}{Построение путей в~цикле}
\begin{programlisting}%
drawarrow (-.25cm, 0)--(10cm, 0)\par
~~~~withpen thinpen;\par
drawarrow (0, -1.25cm)--(0, 1.25cm)\par
~~~~withpen thinpen;\par
\leavevmode\par
draw\par
~~~~((0, 1) for i:=0 step .025 until 9.5:\par
~~~~~~~~--(i, cosd(i**2*rad))\par
~~~~endfor)\par
~~~~scaled cm withpen boldpen;\par
\leavevmode\par
label.bot(btex \$x\$ etex, (10cm, 0));\par
label.lft(btex \$y\$ etex, (0, 1.25cm));
\end{programlisting}
\end{frame}

\begin{frame}{Построение путей в~цикле}
В~приведённом примере предполагается, что определены переменные типа
\literal{pen}~— \literal{thinpen} и~\literal{boldpen}, хранящие перья, тонкое
и~толстое соответственно.

В~числовой переменной \literal{rad} содержится количество градусов в~одном
радиане, т.~е. $\slashfrac{180}\pi\approx57{,}3$.
\end{frame}

\begin{frame}{Условная конструкция}
\begin{flushleft}
\LARGE
\literal{if~}\replaceable{условие}\literal{:~}\replaceable{код}\\
\only<3->{\literal{elseif~}\replaceable{условие}\literal{:~}\replaceable{код}\\
\replaceable{…}\\
\literal{elseif~}\replaceable{условие}\literal{:~}\replaceable{код}\\}
\only<2->{\literal{else:~}\replaceable{код}\\}
\literal{fi}
\end{flushleft}
\only<4>{В~качестве \replaceable{условия} может выступать любое логическое
выражение.}
\end{frame}

\begin{frame}{Примеры логических выражений}
\begin{center}
\begin{tabular}{rl}
$x$\literal{~=~}$y$&$x$ равно $y$\\
$x$\literal{~<>~}$y$&$x$ не равно $y$\\
$x$\literal{~<~}$y$&$x$ меньше $y$\\
$x$\literal{~<=~}$y$&$x$ меньше либо равно $y$\\
$x$\literal{~>~}$y$&$x$ больше $y$\\
$x$\literal{~>=~}$y$&$x$ больше либо равно $y$\\
\literal{not~}$q$&не $q$\\
$p$\literal{~and~}$q$&$p$ и $q$\\
$p$\literal{~or~}$q$&$p$ или $q$\\
\literal{cycle~}$p$&путь $p$ циклический\\
\literal{color~}$v$&значение $v$ имеет тип \literal{color}
\end{tabular}
\end{center}
\end{frame}

\begin{frame}{Цикл \literal{forever} и~оператор \literal{exitif}}
В~\hologo{METAPOST} имеется конструкция безусловного цикла:
\begin{flushleft}
\Large
\literal{forever:}\\
~~~~\replaceable{код}\\
\literal{endfor}
\end{flushleft}

Такой цикл имеет смысл только в~сочетании с~возможностью его прервать при
выполнении определённого условия. Для досрочного прерывания цикла, в~том числе
безусловного, служит оператор
\begin{flushleft}
\Large
\literal{exitif~}\replaceable{условие}
\end{flushleft}
\end{frame}
