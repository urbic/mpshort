\sectionwithlogo{Пути}{\scalebox{.8}{\leavevmode
\begin{mplibcode}
beginfig(0)

path p, q;

p=halfcircle scaled -1.5cm
		..halfcircle scaled 3cm shifted (1.5cm, .75cm)
		..cycle;

for t:=0 step .1 until length p:
	drawarrow (point t of p)--((point t of p)+(unitvector(direction t of p)*1.5cm))
			withcolor .5red withpen tinypen;
endfor

draw p withpen boldpen;

endfig
\end{mplibcode}
}}

\begin{frame}{Полином Бернштейна}
Задаётся $n+1$~точкой $z_0$, $z_1$,~$\ldots$, $z_n$
(\emph{контрольные точки}).

\emph{Полином Бернштейна\/} $n$"=й степени:
	\[
	z(t)=\sum_{i=0}^n\binom niz_i(1-t)^it^{n-i}.
	\]

Биномиальный коэффициент:
	\[
	\binom ni=\frac{n!}{i!(n-i)!}.
	\]

Назван в~честь советского математика Сергея Бернштейна.
\end{frame}

\begin{frame}{Кривые Безье}
Кривые, заданные параметрически полиномом Бернштейна, называются \emph{кривыми
Безье\/} соответствующей степени.

Активно используются в~компьютерной графике для представления линий.

В~\hologo{METAPOST} применяются кривые Безье $3$"=й степени.

Названы в~честь французского инженера Пьера Безье (Piere Étienne Bésier),
указавшего на их важность в~компьютерной графике.
\end{frame}

\begin{frame}{Свойства кривой Безье}
\begin{itemize}
\item
выходит из точки~$z_0$ при~$t=0$
\item
приходит в~точку~$z_n$ при~$t=1$
\item
касательная в~точке~$z_0$ «смотрит» в~точку~$z_1$
\item
касательная в~точке~$z_n$ «смотрит» из точки~$z_{n-1}$
\item
при~$t\in[0;1]$ целиком содержится в~выпуклой оболочке всех контрольных точек
\end{itemize}
\end{frame}

\begin{frame}{Кривая Безье $0$"=й степени}
	\[
	z(t)\equiv z_0
	\]
\begin{center}
\leavevmode
\begin{mplibcode}
beginfig(0)

z1=origin;

markdot.circle(z1) fg=>orange;

label.bot("$z_0$", z1);

endfig
\end{mplibcode}
\\[4ex]
точка
\end{center}
\end{frame}

\begin{frame}{Кривая Безье $1$"=й степени}
	\[
	z(t)=z_0(1-t)+z_1t
	\]
\begin{center}
\leavevmode
\begin{mplibcode}
beginfig(0)

z1=origin;
z2=(2cm, 1.5cm);

draw z1--z2 withpen boldpen;

markdot.circle(z1) fg=>orange;
markdot.circle(z2) fg=>orange;

label.llft("$z_0$", z1);
label.urt("$z_1$", z2);

endfig
\end{mplibcode}
\\[4ex]
прямая линия
\end{center}
\end{frame}

\begin{frame}{Кривая Безье $2$"=й степени}
	\[
	z(t)=z_0(1-t)^2+2z_1(1-t)t+z_2t^2
	\]
\begin{center}
\leavevmode
\begin{mplibcode}
beginfig(0)

z1=origin;
z2=(3cm, 1.5cm);
z3=(4cm, -.5cm);

draw z1--z2 withpen tinypen withcolor .5orange;
draw z2--z3 withpen tinypen withcolor .5orange;
draw z1..controls z2 and z2..z3 withpen boldpen;

markdot.circle(z1) fg=>orange;
markdot.circle(z2) fg=>orange;
markdot.circle(z3) fg=>orange;

label.llft("$z_0$", z1);
label.urt("$z_1$", z2);
label.lrt("$z_2$", z3);

endfig
\end{mplibcode}
\\[4ex]
парабола
\end{center}
\end{frame}

\begin{frame}{Кривая Безье $3$"=й степени}
	\[
	z(t)=z_0(1-t)^3+3z_1(1-t)^2t+3z_2(1-t)t^2+z_3t^3
	\]
\begin{center}
\leavevmode
\begin{mplibcode}
beginfig(0)

z1=origin;
z2=(1cm, 1.5cm);
z3=(4cm, 2.5cm);
z4=(5cm, -.5cm);

draw z1--z2 withpen tinypen withcolor .5orange;
draw z3--z4 withpen tinypen withcolor .5orange;
draw z1..controls z2 and z3..z4 withpen boldpen;

markdot.circle(z1) fg=>orange;
markdot.circle(z2) fg=>orange;
markdot.circle(z3) fg=>orange;
markdot.circle(z4) fg=>orange;

label.bot("$z_0$", z1);
label.top("$z_1$", z2);
label.top("$z_2$", z3);
label.bot("$z_3$", z4);

endfig
\end{mplibcode}
\\[4ex]
кубическая кривая (в~школе не проходят)
\end{center}
\end{frame}

\begin{frame}{Преимущества кривых Безье $3$"=й степени}
\begin{itemize}
\item
хорошо приближают типичные кривые линии
\item
в~памяти компьютера представляются восьмёркой чисел (координатами четырёх
точек)
\item
строятся с~помощью простого рекурсивного алгоритма
\end{itemize}
\end{frame}

\begin{frame}{Построение точки~$z\bigl(\frac12\bigr)$ для кривой $3$"=й
степени}
\begin{center}
\leavevmode
\begin{mplibcode}
beginfig(0)

z1=origin;
z2=(2cm, 3cm);
z3=(8cm, 5cm);
z4=(10cm, -cm);
z.a=.5[z1, z2];
z.b=.5[z2, z3];
z.c=.5[z3, z4];
z.ab=.5[z.a, z.b];
z.bc=.5[z.b, z.c];
z.abbc=.5[z.ab, z.bc];

draw z1--z2--z3--z4 withpen tinypen withcolor .5orange;
draw z.a--z.b--z.c withpen tinypen withcolor .5orange;
draw z.ab--z.bc withpen tinypen withcolor .5orange;
draw z1..controls z2 and z3..z4 withpen boldpen;

markdot.circle(z1) fg=>orange;
markdot.circle(z2) fg=>orange;
markdot.circle(z3) fg=>orange;
markdot.circle(z4) fg=>orange;
markdot.circle(z.a) fg=>orange;
markdot.circle(z.b) fg=>orange;
markdot.circle(z.c) fg=>orange;
markdot.circle(z.ab) fg=>orange;
markdot.circle(z.bc) fg=>orange;
markdot.circle(z.abbc) fg=>red;

label.bot("$z_0$", z1);
label.top("$z_1$", z2);
label.top("$z_2$", z3);
label.bot("$z_3$", z4);
label.ulft("$z_{01}$", z.a);
label.top("$z_{12}$", z.b);
label.urt("$z_{23}$", z.c);
label.top("$z_{0112}$", z.ab);
label.top("$z_{1223}$", z.bc);
label.bot("$z\bigl(\frac12\bigr)$", z.abbc) withcolor red;

endfig
\end{mplibcode}

\end{center}
\end{frame}

\begin{frame}{Построение кривых Безье $3$"=й степени}
Точка $z\bigl(\frac12\bigr)$ строится, как показано на предыдущем слайде.
Отрезок кривой при $t\in\bigl[0;\frac12\bigr]$, оказывается,~— это кривая
с~контрольными точками $z_0$, $z_{01}$, $z_{0112}$, $z\bigl(\frac12\bigr)$.
А~отрезок при $t\in\bigl[\frac12;1\bigr]$~— кривая с~контрольными точками
$z\bigl(\frac12\bigr)$, $z_{1223}$, $z_{23}$, $z_3$.

Для этих двух отрезков также находятся точки, отвечающие $t=\frac12$. Эти точки
разбивают каждый из отрезков на половинки. Для них опять находятся «серединные»
точки. Когда «серединных» точек будет достаточно много, они вычертят линию.
\end{frame}

\begin{frame}{Примеры кривых Безье $3$"=й степени}
\begin{center}
\leavevmode
\begin{mplibcode}
beginfig(0)

z1=origin;
z2=(0, 1.5cm);
z3=(2cm, 1.5cm);
z4=(2cm, 0);

draw z1--z2 withpen tinypen withcolor .5orange;
draw z3--z4 withpen tinypen withcolor .5orange;
draw z1..controls z2 and z3..z4 withpen boldpen;

markdot.circle(z1) fg=>orange;
markdot.circle(z2) fg=>orange;
markdot.circle(z3) fg=>orange;
markdot.circle(z4) fg=>orange;

label.bot("$z_0$", z1);
label.top("$z_1$", z2);
label.top("$z_2$", z3);
label.bot("$z_3$", z4);

endfig
\end{mplibcode}
\hfill
\leavevmode
\begin{mplibcode}
beginfig(0)

z1=origin;
z3=(0, 1.5cm);
z2=(2cm, 1.5cm);
z4=(2cm, 0);

draw z1--z2 withpen tinypen withcolor .5orange;
draw z3--z4 withpen tinypen withcolor .5orange;
draw z1..controls z2 and z3..z4 withpen boldpen;

markdot.circle(z1) fg=>orange;
markdot.circle(z2) fg=>orange;
markdot.circle(z3) fg=>orange;
markdot.circle(z4) fg=>orange;

label.bot("$z_0$", z1);
label.top("$z_1$", z2);
label.top("$z_2$", z3);
label.bot("$z_3$", z4);

endfig
\end{mplibcode}
\hfill
\leavevmode
\begin{mplibcode}
beginfig(0)

z1=origin;
z2=(1.5cm, 0);
z3=(-cm, cm);
z4=(2cm, 0);

draw z1--z2 withpen tinypen withcolor .5orange;
draw z3--z4 withpen tinypen withcolor .5orange;
draw z1..controls z2 and z3..z4 withpen boldpen;

markdot.circle(z1) fg=>orange;
markdot.circle(z2) fg=>orange;
markdot.circle(z3) fg=>orange;
markdot.circle(z4) fg=>orange;

label.bot("$z_0$", z1);
label.bot("$z_1$", z2);
label.top("$z_2$", z3);
label.bot("$z_3$", z4);

endfig
\end{mplibcode}
\\[4ex]
\leavevmode
\begin{mplibcode}
beginfig(0)

z1=origin;
z2=(0, 3cm);
z3=(2cm, 1.5cm);
z4=(2cm, 0);

draw z1--z2 withpen tinypen withcolor .5orange;
draw z3--z4 withpen tinypen withcolor .5orange;
draw z1..controls z2 and z3..z4 withpen boldpen;

markdot.circle(z1) fg=>orange;
markdot.circle(z2) fg=>orange;
markdot.circle(z3) fg=>orange;
markdot.circle(z4) fg=>orange;

label.bot("$z_0$", z1);
label.top("$z_1$", z2);
label.top("$z_2$", z3);
label.bot("$z_3$", z4);

endfig
\end{mplibcode}
\hfill
\leavevmode
\begin{mplibcode}
beginfig(0)

z1=origin;
z2=(0, 1.5cm);
z3=(2cm, 0);
z4=(2cm, 3cm);

draw z1--z2 withpen tinypen withcolor .5orange;
draw z3--z4 withpen tinypen withcolor .5orange;
draw z1..controls z2 and z3..z4 withpen boldpen;

markdot.circle(z1) fg=>orange;
markdot.circle(z2) fg=>orange;
markdot.circle(z3) fg=>orange;
markdot.circle(z4) fg=>orange;

label.bot("$z_0$", z1);
label.top("$z_1$", z2);
label.bot("$z_2$", z3);
label.top("$z_3$", z4);

endfig
\end{mplibcode}
\hfill
\leavevmode
\begin{mplibcode}
beginfig(0)

z1=origin;
z3=(0, 1.5cm);
z2=(4cm, .5cm);
z4=(2cm, 0);

draw z1--z2 withpen tinypen withcolor .5orange;
draw z3--z4 withpen tinypen withcolor .5orange;
draw z1..controls z2 and z3..z4 withpen boldpen;

markdot.circle(z1) fg=>orange;
markdot.circle(z2) fg=>orange;
markdot.circle(z3) fg=>orange;
markdot.circle(z4) fg=>orange;

label.bot("$z_0$", z1);
label.top("$z_1$", z2);
label.top("$z_2$", z3);
label.bot("$z_3$", z4);

endfig
\end{mplibcode}

\end{center}
\end{frame}

\begin{frame}{Пути в~\hologo{METAPOST}}
\begin{columns}
\column{.4\textwidth}
\only<1>{\leavevmode
\begin{mplibcode}
input snake;

beginfig(0)

draw snake withpen boldpen;

endfig
\end{mplibcode}
}
\only<2>{\leavevmode
\begin{mplibcode}
input snake;

beginfig(0)

draw snake withpen boldpen;
showseg(snake, 0);

endfig
\end{mplibcode}
}
\only<3>{\leavevmode
\begin{mplibcode}
input snake;

beginfig(0)

draw snake withpen boldpen;
showseg(snake, 1);

endfig
\end{mplibcode}
}
\only<4>{\leavevmode
\begin{mplibcode}
input snake;

beginfig(0)

draw snake withpen boldpen;
showseg(snake, 2);

endfig
\end{mplibcode}
}
\only<5>{\leavevmode
\begin{mplibcode}
input snake;

beginfig(0)

draw snake withpen boldpen;
showseg(snake, 3);

endfig
\end{mplibcode}
}
\column{.6\textwidth}
Пути в~\hologo{METAPOST} составлены из сегментов~— кривых Безье:\\
\bigskip
\pause

{\Large
\only<2>{первый}%
\only<3>{второй}%
\only<4>{третий}%
\only<5>{четвёртый}%
\ сегмент}
\end{columns}
\end{frame}

\begin{frame}{Использование путей в~\hologo{METAPOST}}
\begin{columns}
\column{.4\textwidth}
\begin{center}
\only<1>{\leavevmode
\begin{mplibcode}
beginfig(0)

draw halfcircle scaled -1.5cm
		..halfcircle scaled 3cm shifted (1.5cm, .75cm)
		..cycle withpen boldpen;
endfig
\end{mplibcode}
}%
\only<2>{\leavevmode
\begin{mplibcode}
beginfig(0)

fill halfcircle scaled -1.5cm
		..halfcircle scaled 3cm shifted (1.5cm, .75cm)
		..cycle withpen boldpen;
endfig
\end{mplibcode}
}%
\only<3>{\leavevmode
\begin{mplibcode}
beginfig(0)

for i:=0 upto 50:
	draw fullcircle scaled (.2cm*i) shifted (.5cm, .5cm)
			withcolor orange withpen boldpen;
endfor

clip currentpicture to halfcircle scaled -1.5cm
		..halfcircle scaled 3cm shifted (1.5cm, .75cm)
		..cycle;
endfig
\end{mplibcode}
}%
\end{center}
\column{.6\textwidth}
Пути можно:
\begin{itemize}
\item\alert<1>{начертить}
\item\alert<2>{закрасить}
\item\alert<3>{использовать как границу рисунка (to clip)}
\transdissolve<2-3>
\end{itemize}
\end{columns}
\begin{center}
\uncover<2-3>{\alert{только \emph{циклические}!}}
\end{center}
\end{frame}

\begin{frame}{Циклические пути}
Путь называется \emph{замкнутым}, если его начальная и~конечная точки
совпадают.

Замкнутый путь, имеющий атрибут цикличности, называется \emph{циклическим}.

Лишь циклические пути могут быть закрашены или использованы как граница
рисунка.

Атрибут цикличности устанавливается при создании пути.
\end{frame}

\begin{frame}{Число вращения}
\only<1>{%
Для замкнутых путей определяется \emph{число вращения}.

Это количество полных оборотов, которые сделает единичный касательный вектор,
если точка касания полностью обойдёт путь.

}
\begin{columns}
\column{.6\textwidth}
\begin{center}
\only<2>{\leavevmode
\begin{mplibcode}
beginfig(0)

path p, q;

p=halfcircle scaled -1.5cm
		..halfcircle scaled 3cm shifted (1.5cm, .75cm)
		..cycle;

for t:=0 step .1 until length p:
	drawarrow (point t of p)--((point t of p)+(unitvector(direction t of p)*1.5cm))
			withcolor .5red withpen tinypen;
endfor

draw p withpen boldpen;

endfig
\end{mplibcode}
}%
\only<3>{\leavevmode
\begin{mplibcode}
beginfig(0)

path p, q;

p=reverse(halfcircle scaled -1.5cm
		..reverse halfcircle scaled 3cm shifted (1.5cm, .75cm)
		..cycle);

for t:=0 step .1 until length p:
	drawarrow (point t of p)--((point t of p)+(unitvector(direction t of p)*1.5cm))
			withcolor .5red withpen tinypen;
endfor

draw p withpen boldpen;

endfig
\end{mplibcode}
}%
\only<4>{\leavevmode
\begin{mplibcode}
beginfig(0)

path p, q;

p=halfcircle scaled 1.5cm
		..halfcircle scaled 3cm shifted (.25cm, .25cm)
		..cycle;

for t:=0 step .1 until length p:
	drawarrow (point t of p)--((point t of p)+(unitvector(direction t of p)*1.5cm))
			withcolor .5red withpen tinypen;
endfor

draw p withpen boldpen;

endfig
\end{mplibcode}
}%
\end{center}
\column{.4\textwidth}
\only<2-4>{\large число вращения \alert{\only<2>{1}\only<3>{0}\only<4>{2}}}%
\end{columns}
\transdissolve<3-4>
\end{frame}

\begin{frame}{Предопределённые переменные~— пути}
\begin{columns}
\column{.5\textwidth}
\begin{center}
\only<1>{\leavevmode
\begin{mplibcode}
beginfig(0)

drawaxes(-1.5cm, -.5cm, 1.5cm, 1.5cm);

draw quartercircle scaled 2cm withpen boldpen;

label.bot("$\frac12$", cm*right);
label.lft("$\frac12$", cm*up);

endfig
\end{mplibcode}
}%
\only<2>{\leavevmode
\begin{mplibcode}
beginfig(0)

drawaxes(-1.5cm, -.5cm, 1.5cm, 1.5cm);

draw halfcircle scaled 2cm withpen boldpen;

label.bot("$\frac12$", cm*right);
label.ulft("$\frac12$", cm*up);

endfig
\end{mplibcode}
}%
\only<3>{\leavevmode
\begin{mplibcode}
beginfig(0)

drawaxes(-1.5cm, -1.5cm, 1.5cm, 1.5cm);

draw fullcircle scaled 2cm withpen boldpen;

label.lrt("$\frac12$", cm*right);
label.ulft("$\frac12$", cm*up);

endfig
\end{mplibcode}
}%
\only<4>{\leavevmode
\begin{mplibcode}
beginfig(0)

drawaxes(-.5cm, -.5cm, 2.5cm, 2.5cm);

draw unitsquare scaled 2cm withpen boldpen;

label.bot("$1$", 2cm*right);
label.lft("$1$", 2cm*up);

endfig
\end{mplibcode}
}%
\end{center}
\column{.5\textwidth}
{\LARGE\literal{%
\only<1>{quartercircle}%
\only<2>{halfcircle}%
\only<3>{fullcircle}%
\only<4>{unitsquare}%
}}%
\end{columns}
\bigskip
\begin{center}
\only<1>{четверть окружности радиуса~$\frac12$}%
\only<2>{половина окружности радиуса~$\frac12$}%
\only<3>{окружность радиуса~$\frac12$}%
\only<1-3>{;\\центр в~начале координат}%
\only<4>{единичный квадрат;\\левый нижний угол в~начале координат}%
\only<3,4>{\\этот путь циклический}%
\end{center}
\end{frame}

\begin{frame}{Ломаные}
\begin{columns}
\column{.4\textwidth}
\begin{center}
\only<1>{\leavevmode
\begin{mplibcode}
beginfig(0)

drawaxes(-.5cm, -.5cm, 3cm, 2cm);
draw origin--(2.5cm, 1.5cm) withpen boldpen;

endfig
\end{mplibcode}
}%
\only<2>{\leavevmode
\begin{mplibcode}
beginfig(0)

drawaxes(-.5cm, -.5cm, 3cm, 2cm);
draw origin--(2.5cm, 1.5cm)--(2.5cm, 0) withpen boldpen;

endfig
\end{mplibcode}
}%
\only<3>{\leavevmode
\begin{mplibcode}
beginfig(0)

drawaxes(-.5cm, -.5cm, 3cm, 2.5cm);
draw origin--(2.5cm, 1.5cm)--(2.5cm, 0)--(cm, 2cm) withpen boldpen;

endfig
\end{mplibcode}
}%
\only<4>{\leavevmode
\begin{mplibcode}
beginfig(0)

drawaxes(-.5cm, -.5cm, 3cm, 2.5cm);
draw origin--(2.5cm, 1.5cm)--(2.5cm, 0)--(cm, 2cm)--cycle withpen boldpen;

endfig
\end{mplibcode}
}
\end{center}
\column{.6\textwidth}
\begin{programlisting}
\only<1>{origin\par
~~~~--(2.5cm, 1.5cm)}%
\only<2>{origin\par
~~~~--(2.5cm, 1.5cm)\par
~~~~--(2.5cm, 0)}%
\only<3>{origin\par
~~~~--(2.5cm, 1.5cm)\par
~~~~--(2.5cm, 0)\par
~~~~--(cm, 2cm)}%
\only<4>{origin\par
~~~~--(2.5cm, 1.5cm)\par
~~~~--(2.5cm, 0)\par
~~~~--(cm, 2cm)\par
~~~~\alert{--cycle}}
\end{programlisting}
\end{columns}
\end{frame}

\begin{frame}{Связка путей «\literal{--}»}
Из путей~$p_1$ и~$p_2$ образуется новый путь: он состоит из пути~$p_1$,
\alert{прямолинейного} отрезка, соединяющего конец~$p_1$ с~началом~$p_2$,
и~пути~$p_2$.
\bigskip
\begin{columns}
\column{.5\textwidth}
\begin{center}
\only<1>{\leavevmode
\begin{mplibcode}
beginfig(0)

draw halfcircle scaled -cm withpen boldpen;
draw halfcircle scaled 2cm shifted (cm, .5cm) withpen boldpen;

endfig
\end{mplibcode}
}%
\only<2>{\leavevmode
\begin{mplibcode}
beginfig(0)

draw halfcircle scaled -cm
		--halfcircle scaled 2cm shifted (1cm, .5cm) withpen boldpen;

endfig
\end{mplibcode}
}
\only<3>{\leavevmode
\begin{mplibcode}
beginfig(0)

draw halfcircle scaled -cm
		--halfcircle scaled 2cm shifted (1cm, .5cm)
		--cycle withpen boldpen;

endfig
\end{mplibcode}
}
\end{center}
\column{.5\textwidth}
\begin{center}
\begin{programlisting}
\only<1>{p2\par
p1}%
\only<2>{p1--p2}%
\only<3>{p1--p2--cycle}%
\end{programlisting}
\end{center}
\end{columns}
\end{frame}

\begin{frame}{Связка путей «\literal{..}»}
Из путей~$p_1$ и~$p_2$ образуется новый путь: он состоит из пути~$p_1$,
\alert{криволинейного} отрезка, \alert{гладко} соединяющего конец~$p_1$
с~началом~$p_2$, и пути~$p_2$.

Детали того, как строится криволинейная связка, довольно сложны.
\bigskip
\begin{columns}
\column{.5\textwidth}
\begin{center}
\only<1>{\leavevmode
\begin{mplibcode}
beginfig(0)

draw halfcircle scaled -cm withpen boldpen;
draw halfcircle scaled 2cm shifted (cm, .5cm) withpen boldpen;

endfig
\end{mplibcode}
}%
\only<2>{\leavevmode
\begin{mplibcode}
beginfig(0)

draw halfcircle scaled -cm
		..halfcircle scaled 2cm shifted (1cm, .5cm) withpen boldpen;

endfig
\end{mplibcode}
}
\only<3>{\leavevmode
\begin{mplibcode}
beginfig(0)

draw halfcircle scaled -cm
		..halfcircle scaled 2cm shifted (1cm, .5cm)
		..cycle withpen boldpen;

endfig
\end{mplibcode}
}
\end{center}
\column{.5\textwidth}
\begin{center}
\begin{programlisting}
\only<1>{p2\par
p1}%
\only<2>{p1..p2}%
\only<3>{p1..p2..cycle}%
\end{programlisting}
\end{center}
\end{columns}
\end{frame}

\begin{frame}{Связка путей «\literal{\&}»}
Из путей~$p_1$ и~$p_2$ образуется новый путь: он получается присоединением
$p_2$ к~$p_1$.

Конец~$p_1$ должен совпадать с~началом~$p_2$, иначе ошибка:
\begin{screen}
\only<1>{! Paths don't touch; `\&' will be changed to `..'.}%
\only<2>{! Пути не соединяются; `\&' будет заменён на `..'.}%
\end{screen}
\hologo{METAPOST} заменит опасную связку «\literal{\&}» на безопасную
«\literal{..}».
\transdissolve<2>
\end{frame}

\begin{frame}{Указание направлений}
\begin{columns}
\column{.4\textwidth}
\only<1>{\leavevmode
\begin{mplibcode}
beginfig(0)

z1=origin;
z2=(3.5cm, 0);

draw z1{dir -30}..{dir -45}z2 withpen boldpen;

label.top("$z_1$", z1);
label.bot("$z_2$", z2);

endfig
\end{mplibcode}
}%
\only<2>{\leavevmode
\begin{mplibcode}
beginfig(0)

z1=origin;
z2=(3.5cm, 0);

draw z1{dir 0}..{dir -45}z2 withpen boldpen;

label.bot("$z_1$", z1);
label.bot("$z_2$", z2);

endfig
\end{mplibcode}
}%
\only<3>{\leavevmode
\begin{mplibcode}
beginfig(0)

z1=origin;
z2=(3.5cm, 0);

draw z1{dir 30}..{dir -45}z2 withpen boldpen;

label.bot("$z_1$", z1);
label.bot("$z_2$", z2);

endfig
\end{mplibcode}
}%
\only<4>{\leavevmode
\begin{mplibcode}
beginfig(0)

z1=origin;
z2=(3.5cm, 0);

draw z1{dir 60}..{dir -45}z2 withpen boldpen;

label.bot("$z_1$", z1);
label.bot("$z_2$", z2);

endfig
\end{mplibcode}
}%
\only<5>{\leavevmode
\begin{mplibcode}
beginfig(0)

z1=origin;
z2=(3.5cm, 0);

draw z1{dir 90}..{dir -45}z2 withpen boldpen;

label.bot("$z_1$", z1);
label.bot("$z_2$", z2);

endfig
\end{mplibcode}
}%
\column{.6\textwidth}
\begin{programlisting}
z1\{dir
\alert{\only<1>{-30}%
\only<2>{0}%
\only<3>{30}%
\only<4>{60}%
\only<5>{90}%
}\}\par
~~~~..\{dir -45\}z2
\end{programlisting}
\end{columns}
\end{frame}

\begin{frame}{Параметрическая длина пути}
Оператор \literal{length} вычисляет так называемую параметрическую длину. Это
целое число~— количество сегментов кривых Безье, из которых составлен путь:
\begin{center}
\Large
\literal{length}~$p$
\end{center}
\end{frame}

\begin{frame}{Параметризация пути}
Каждому числовому значению~$t\in[0;l]$, где~$l$~— параметрическая длина
пути~$p$, можно поставить в~соответствие точку на этом пути.

Пусть~$\alpha=[t]$ (целая часть~$t$), $\beta=\{t\}$ (дробная часть~$t$).

Возьмём сегмент~$p_\alpha$ кривой Безье пути~$p$, имеющий
номер~$\alpha$. Обозначим как $z_{0,1,2,3}$ контрольные точки
этого сегмента. Возьмём точку, заданную значением~$\beta$ в~полиноме Бернштейна
для этого сегмента:
	\[
	z(\beta)=(1-\beta)^3z_0+3(1-\beta)^2\beta z_1+3(1-\beta)\beta^2z_1+\beta^3z_3.
	\]
Полученная точка~$z(\beta)$ и~есть искомая точка на пути~$p$, отвечающая
значению параметра~$t$.
\end{frame}

\begin{frame}{Оператор \literal{point of}}
Оператор \literal{point of} делает все эти вычисления:
\begin{center}
\Large
\literal{point~}$t$\literal{~of~}$p$
\end{center}

Значения~$t$, расположенные вне отрезка~$[0;l]$, сдвигаются в~нужную сторону,
пока не окажутся в~пределах этого отрезка.

Таким образом, \literal{point -1 of~}$p$ даст тот~же результат, что
и~\literal{point 0 of~}$p$.

\literal{point infinity of~}$p$~— это всегда конечная точка пути.
(\literal{infinity}~— предопределённая числовая переменная, самое
большое число, представимое в~\hologo{METAPOST}.)
\end{frame}

\begin{frame}{Оператор \literal{point of}}
\begin{columns}
\column{.4\textwidth}
\leavevmode
\begin{mplibcode}
beginfig(0)

path c;
c=fullcircle scaled 4cm;

draw c withpen boldpen;

for i:=0 step .5 until length c:
	markdot.circle(point i of c) fg=>orange;
endfor

endfig
\end{mplibcode}

\column{.6\textwidth}
На рисунке на пути $p={}$\literal{fullcircle} расставлены отметки в~точках
\literal{point~}$t$\literal{~of~}$p$ с~шагом~$\frac14$.

\bigskip
Если приглядеться, видно, что отметки расположены не совсем равномерно.
\end{columns}
\end{frame}

\begin{frame}{Пересечение путей}
Оператор \literal{intersectionpoint} вычисляет точку пересечения двух путей
$p_1$~и~$p_2$:
\begin{center}
\Large
$p_1$ \literal{intersectionpoint} $p_2$
\end{center}

Если пути не пересекаются, возникает ошибка:
\begin{screen}
\only<1>{! The paths don't intersect.}%
\only<2>{! Пути не пересекаются.}%
\transdissolve<2>
\end{screen}
\end{frame}

\begin{frame}{Пересечение путей}
Если точек пересечения путей $p_1$~и~$p_2$ несколько, вычисляется
по~возможности та из~них, которая лежит ближе к~началам каждого из~путей.

Поскольку это пожелание по~своей сути противоречиво, алгоритм поиска точки
пересечения путей определённым образом упорядочивает точки пересечения,
и~выдаёт первую из~них (в~смысле этого упорядочения).

Принцип упорядочения не самый очевидный, а~детали мы опустим.

Вообще говоря, оператор \literal{intersectionpoint} не коммутативен.
\end{frame}
